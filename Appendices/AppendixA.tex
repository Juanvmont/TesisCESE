% Appendix A

\chapter{Manual de Usuario} % Main appendix title

\label{AppendixA} % For referencing this appendix elsewhere, use \ref{AppendixA}
Para implementar el "Sistema de Monitoreo de Salud de Nodos WSN Alimentados a Energía Solar" en el \textit{Mote LSE} es necesario establecer la conexión según el siguiente diagrama:
\vspace{10px}
\begin{figure}[h!]
	\centering
    \includegraphics[width=0.9\textwidth]{./Figures/conex.png}
    	\caption{Diagrama de Conexión.}
	\label{fig:conex}
\end{figure}

Se deben seguir los siguientes pasos:
\begin{itemize}
	\item Descargar e instalar la versión v8 del LPCxpresso IDE adecuado según el sistema operativo de la estación de trabajo (Ver sección Downloads de \citep{IDE}).
	\item Para completar el toolchain se hace uso del programador y debugger \textit{(LPC-Link)} conectado al \textit{Mote LSE}. (Ver detalle de los pines en la figura \ref{fig:10pin}).
	\item Para abrir el workspace del proyecto, obtener el archivo ws-wsn.rar y copiar el contenido del archivo descomprimido a su directorio de elección.
	\item En el entorno de desarrollo, ir a File->Import, esto abre la ventana de importación.
	\item Expandir la opción “General”, seleccionar “Existing projects into Workspace” y presionar “Next”.
	\item Presionar “Browse” y seleccionar el directorio elegido para el ws, presionar “OK” En la ventana “Project Window” aparecerán varios proyectos, seleccionarlos todos con “select all”.
	\item No seleccionar el cuadro “Copy projects into workspace”, porque los proyectos tienen paths relativos que se romperían al copiarse al workspace.
	\item Presionar “Finish”, todos los proyectos del workspace deberían estar disponibles en la perspectiva “Project explorer” del entorno de desarrollo.
	\item Para compilar el proyecto, coordinador o device, seleccionarlo en la perspectiva “Project Explorer”, en la ventana “Quick Start Window” aparecerán las opciones “Build”, “Clean”, “Debug” y “Edit project settings”.
	\item Presionar primero “Build all projects” y luego en “Debug” en la perspectiva “QuickStart”. En la consola se podrán ver los mensajes durante el compilado de las librerías y la aplicación.
	\item Seleccionar el emulador “LPC-Link Probe v1.3” y presionar “OK”. En caso que no reconozca el emulador, seleccionar "Search for any enabled emulator".
	\item Para cambiar los parámetros, modificarlos según las siguientes tablas (\ref{tab:paracoo} y \ref{tab:paradevi})y repetir los ultimos 3 pasos.
	\end{itemize}

\begin{figure}[h!]
	\centering
    \includegraphics[width=.5\textwidth]{./Figures/pin.png}
    	\caption{Conexión 10-Pin de ARM}
	\label{fig:10pin}
\end{figure}

Para configurar el nodo como coordinador, configurar los siguientes parámetros:
\begin{table}[ht]
	\centering
	\caption{Descripción del byte de Estado de Operación}
	\begin{tabular}{@{} l *1c @{}}    \toprule
		\emph{\textbf{Valor}} & \emph{\textbf{Descripción}}\\
		\midrule
		00000000 &  Modo Batería\\
		11111111 &  Modo Panel\\
		Otros & Reservado\\
		\bottomrule
		\hline
	\end{tabular}
	\label{tab:paracoo}
\end{table}

Para configurar el nodo como dispositivo, configurar los siguientes parámetros:
\begin{table}[ht]
	\centering
	\caption{Descripción del byte de Estado de Operación}
	\begin{tabular}{@{} l *1c @{}}    \toprule
		\emph{\textbf{Valor}} & \emph{\textbf{Descripción}}\\
		\midrule
		00000000 &  Modo Batería\\
		11111111 &  Modo Panel\\
		Otros & Reservado\\
		\bottomrule
		\hline
	\end{tabular}
	\label{tab:paradevi}
\end{table}

