% Chapter Template

\chapter{Conclusiones} % Main chapter title

\label{Chapter5} % Change X to a consecutive number; for referencing this chapter elsewhere, use \ref{ChapterX}


%----------------------------------------------------------------------------------------

%----------------------------------------------------------------------------------------
%	SECTION 1
%----------------------------------------------------------------------------------------

\section{Conclusiones del trabajo realizado}

En la presente memoria de tesis se ha documentado el Trabajo Final para la obtención del grado de \degreename.  

Se obtuvo un \texttt{firmware} que permite monitoriar y controlar un acuario en forma remota mediante una \keyword{interfaz web} embebida en la plataforma \keyword{CIAA-NXP}. Se pudo modelar el comportamiento de un acuario real mediante una ``planta piloto'' para poder avanzar en el desarrollo del \texttt{firmware} de control.

Se pudo integrar el stack TCP/IP \keyword{\texttt{lwIP}} con el sistema operativo de tiempo real \keyword{\texttt{freeRTOS}} al firmware de control sobre la base de un \texttt{port} desarrollado por NXP, fabricante del microcontrolador.

\medskip

Durante el desarrollo de este Trabajo Final se aplicaron conocimientos adquiridos a lo largo de la Carrera de Especialización en Sistemas Embebidos.  Si bien el conjunto total de asignaturas cursadas  aportaron conocimientos necesarios para la práctica profesional en el área de los Sistemas Embebidos, se quiere dejar constancia en particular, de las asignaturas con mayor relevancia para el trabajo presentado.

\begin{itemize}
\item
\textbf{Arquitectura de microprocesadores}. Resultó necesario tener conocimientos sobre la Arquitectura ARM Cortex M para la programación de la plataforma \keyword{CIAA-NXP} y el uso de los periféricos.  \textit{Algo del mapa de memoria. Algo de las interrupciones.}

\item
\textbf{Programación de microprocesadores}. Se utilizaron buenas prácticas de programación en Lenguaje C para microcontroladores y periféricos, aprendidas en esta asignatura. Se usó un formato del código consitente: comentarios en las declaraciones de funciones y partes importantes del código, constantes en mayúsculas, \texttt{camelCase} para poner nombres significativos a funciones y variables. Se utilizaron APIs para abstraer distintas capas del código. Se obtuvo un código más legible y reutilizable que puede ser portado a distintas plataformas con menor esfuerzo.

\item
\textbf{Ingeniería de Software en Sistemas Embebidos}. Se utilizaron técnicas provinientes de la Ingeniería de Software. Siempre que fue posible se utilizó un método sistemático e iterativo de desarrollo pensando en el ciclo de vida del \texttt{software}. Se hizo uso extensivo de \texttt{git} como sistema de control de versiones distribuido del \texttt{software}.  \textit{Algo de diagramas UML.}

\item
\textbf{Gestión de Proyectos en Ingeniería}. Resultó de mucha utilidad elaborar un Plan de Proyecto para organizar el Trabajo Final.  Parte del material elaborado en esta asignatura se encuentra en la presente memoria. \textit{Definición del proyecto, gestión de tiempos, análisis de riesgos, planificación de tareas.}

\item
\textbf{Sistemas Operativos de Tiempo Real (I y II)}. Se aplicaron los conocimientos adquiridos sobre freeRTOS respecto a tareas, cambios de contexto, manejo de colas, semáforos y mutex entre otras cosas.  Asimismo, se utilizaron herramientas de desarrollo propias de \texttt{freeRTOS} para medir el uso de las pilas de las tareas creadas y optimizar el uso de la memoria.

\item 
\textbf{Protocolos de Comunicación}. Se aplicaron ampliamente los conocimientos obtenidos en el área de \texttt{Ethernet}. En particular, resultó muy útil el material sobre el stack tpc/ip \keyword{\texttt{lwIP}}.

\item
\textbf{Diseño de Sistemas Críticos}. Siempre que fue posible se utilizon técnicas de programación defensiva para que el comportamiento del \texttt{firmware} resulte lo más predecible posible.  Se buscó obtener un sistema de control que no falle en primer lugar, y que si lo hace, falle en forma segura evitando daños a los seres vivos dentro y fuera del acuario y la propiedad. \textit{Algo sobre Utilización de estándares, MISRA C, C99 y buenas prácticas de programación}.
\end{itemize}



\medskip

\noindent Asimismo, se adquirió conocimientos en las áreas de:

\begin{itemize}
\item Programación en \textbf{\texttt{HTML5}}.
\item Programación en \textbf{\texttt{Javascript}}.  
\item Utilización de \textbf{\texttt{linker scripts}} para el entorno de desarrollo \texttt{LPCXpresso}.
\end{itemize}


\medskip

Por lo tanto, se llega a la conclusión que los objetivos planteados al comenzar el trabajo han sido alcanzados satisfactoriamente, y además, se obtienen conocimientos muy importantes para la formación profesional del autor.

%----------------------------------------------------------------------------------------
%	SECTION 2
%----------------------------------------------------------------------------------------

\section{Trabajo futuro}

\textbf{página web:}

portabilidad: versión para distintos dispositivos: teléfonos, tablets, etc

apariencia: diseño escalable para distintas resoluciones de pantalla. 

accesibilidad: integrar mensajes de ayuda describiendo los elementos de interacción con el usuario.

robustez: validación de datos, formularios, etc. 

seguridad: Ingreso con usuario y contraseña a las páginas que permitan hacer cambios en el acuario.


\textbf{Planta piloto:}

Utilización de sensores y actuadores reales.




