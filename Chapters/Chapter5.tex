% Chapter Template

\chapter{Conclusiones} % Main chapter title

\label{Chapter5} % Change X to a consecutive number; for referencing this chapter elsewhere, use \ref{ChapterX}

Se diseñó un \textit{harware} que funcionan como Generador Eléctrico Solar Autónomo (GESA) y un \textit{firmware} que permite controlar a sus dispositivos para evitar que el problema se manifieste y si eventualmente se manifiesta, detectarlo e informarlo para tomar las acciones necesarias. En conjunto aportan una solución a un problema de disponibilidad de nodos debido a su autonomía energética.

%----------------------------------------------------------------------------------------
%	SECTION 1
%----------------------------------------------------------------------------------------

\section{Lecciones aprendidas}

\medskip
Durante el desarrollo de este trabajo integrador se aplicaron conocimientos adquiridos a lo largo de la Carrera de Especialización en Sistemas Embebidos. Principalmente de las siguientes asignaturas: 

\begin{itemize}
\item
\textbf{Arquitectura de microprocesadores}. Resultó necesario tener conocimientos sobre la Arquitectura ARM Cortex M para la programación de la plataforma \textit{Mote LSE} y su interacción con los periféricos.

\item
\textbf{Programación de microprocesadores}. Se utilizaron los principios de programación en Lenguaje C para microcontroladores y periféricos. Utilizando el principio de diseño KISS\footnote{\textit{del inglés Keep It Simple, Stupid!: «¡Manténlo simple, estúpido!»}} que establece a la simplicidad como objetivo clave de diseño de cualquier sistema y cualquier complejidad innecesaria debe ser evitada. Mas aún cuando el ahorro de energía es un tema crítico para la aplicación y se desea evitar un mayor consumo en la operación. 

\item
\textbf{Gestión de Proyectos en Ingeniería}. El Plan de Proyecto Final, llevó a organizar el trabajo metódicamente. Se emplearon buenas practicas de gestión de proyectos que pueden llevarse a cabo en cualquier ámbito de la ingeniería, tanto laboral como académico.

\item 
\textbf{Protocolos de Comunicación}. Se aplicaron los conocimientos obtenidos sobre los protocolos de comunicación SPI, USB y 802.15.4.

\item 
\textbf{Introducción a WSN}. .
\end{itemize}

\medskip

\noindent Además, durante el desarrollo del trabajo final también se adquirieron conocimientos en:

\begin{itemize}
	\item Sistema de control de versiones de software.
	\item Red eléctrica inteligente (o REI; \textit{Smart Grid} en inglés).
	\item Energías alternativas.	
\end{itemize}

\medskip

Por lo tanto, se concluye que los objetivos planteados al comienzo del trabajo han sido alcanzados satisfactoriamente, cumpliendo con el requerimiento de mas alta prioridad: Presentar un trabajo final con la finalización del proyecto.

%----------------------------------------------------------------------------------------
%	SECTION 2
%----------------------------------------------------------------------------------------
\section{Análisis de costos}

%----------------------------------------------------------------------------------------
%	SECTION 3
%----------------------------------------------------------------------------------------
\section{Implementación de otros algoritmos}
De interés en el área de Sistemas Embebidos
%----------------------------------------------------------------------------------------
%	SECTION 4
%----------------------------------------------------------------------------------------
\section{Próximos pasos}

\begin{itemize}
	\item Hacer la adaptación necesaria del sistema para ser soportado por las plataformas disponibles del Proyecto CIAA (Computadora Industrial Abierta Argentina)y de esta manera brindarle autonomía solar a aplicaciones industriales que no tengan disponibilidad de redes de distribución eléctrica .
\end{itemize}






