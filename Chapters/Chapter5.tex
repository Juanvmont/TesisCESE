% Chapter Template

\chapter{Conclusiones} % Main chapter title

\label{Chapter5} % Change X to a consecutive number; for referencing this chapter elsewhere, use \ref{ChapterX}


%----------------------------------------------------------------------------------------

%----------------------------------------------------------------------------------------
%	SECTION 1
%----------------------------------------------------------------------------------------

\section{Conclusiones del trabajo realizado}

En la presente memoria de tesis se ha documentado el Trabajo Final para la obtención del grado de \degreename.  Se obtuvo una \keyword{interfaz web} embebida en la plataforma \keyword{CIAA-NXP} que permite monitoriar y controlar un acuario en forma remota. Se pudo modelar y maquetar el comportamiento de un acuario real para obtener una planta piloto que posibilitó el desarrollo del \keyword{firmware} de control.

Se pudo integrar el stack TCP/IP \texttt{lwIP} con el sistema operativo de tiempo real \texttt{freeRTOS} al firmware de control sobre la base de un port desarrollado por el fabricante del microcontrolador, NXP.

\medskip

Durante el desarrollo de este Trabajo Final se aplicaron conocimientos adquiridos a lo largo de la Carrera de Especialización en Sistemas Embebidos.  Si bien el conjunto total me asignaturas cursadas  aportó un conocimiento general de la práctica profesional en el área de los Sistemas Embebidos, se quiere dejar constancia en particular, de las asignaturas con mayor relevancia para el trabajo presentado.

\begin{itemize}
\item
\textbf{Arquitectura de microprocesadores}. De esta asignatura se emplean los conocimientos adquiridos sobre la arquitectura ARM Cortex M necesarios para implementar en lenguaje \textit{assembler} las funciones que realizan el cambio de contexto, necesarias para que funcione el concepto de Proceso SCJ. 

\item
\textbf{Programación de microprocesadores}. De esta asignatura se aprovecha la experiencia sobre lenguaje C para microcontroladores de 32 bits y el manejo de sus periféricos. Fue de especial importancia, debido a que; a excepción de unas pocas, todas las funciones para portar HVM a la CIAA debían realizarse en lenguaje C. Este lenguaje también se utilizó en la creación de  la API para el manejo de periféricos.

\item
\textbf{Ingeniería de Software en Sistemas Embebidos}. Se utilizaron técnicas provinientes de la ingeniería de software. Siempre que fue posible se utilizó un método sistemático de desarrollo, para hacer mejor uso de los recursos del proyecto, que aportan calidad y eficiencia la desarrollo. En particular, el diseño iterativo, la utilización de repositorios de software y el diseño modular en capas.

\item
\textbf{Gestión de Proyectos en Ingeniería}. %Durante ésta se desarrolló el Plan de Proyecto del Trabajo Final, permitiendo desde un principio tener una clara planificación del trabajo a realizar. 

\item
\textbf{Sistemas Operativos de Tiempo Real (I y II)}. %De estas asignaturas se aplica el conocimiento obtenido sobre planificadores de tareas expropiativos y la manera en que trabajan. Esto ha sido muy importante para la realización de este Trabajo Final.

\item 
\textbf{Protocolos de Comunicación}.
\item
\textbf{Diseño de Sistemas Críticos}.es industriales.
\end{itemize}



\medskip

\noindent También, se han adquirido aprendizajes en las temáticas:


\medskip

Por lo tanto, se llega a la conclusión que los objetivos planteados al comenzar el trabajo han sido alcanzados satisfactoriamente, y además, se obtienen conocimientos muy importantes para la formación profesional del autor.

%----------------------------------------------------------------------------------------
%	SECTION 2
%----------------------------------------------------------------------------------------

\section{Trabajo futuro}

Sed ullamcorper quam eu nisl interdum at interdum enim egestas. Aliquam placerat justo sed lectus lobortis ut porta nisl porttitor. Vestibulum mi dolor, lacinia molestie gravida at, tempus vitae ligula. Donec eget quam sapien, in viverra eros. Donec pellentesque justo a massa fringilla non vestibulum metus vestibulum. Vestibulum in orci quis felis tempor lacinia. Vivamus ornare ultrices facilisis. Ut hendrerit volutpat vulputate. Morbi condimentum venenatis augue, id porta ipsum vulputate in. Curabitur luctus tempus justo. Vestibulum risus lectus, adipiscing nec condimentum quis, condimentum nec nisl. Aliquam dictum sagittis velit sed iaculis. Morbi tristique augue sit amet nulla pulvinar id facilisis ligula mollis. Nam elit libero, tincidunt ut aliquam at, molestie in quam. Aenean rhoncus vehicula hendrerit.