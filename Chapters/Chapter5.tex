% Chapter Template

\chapter{Conclusiones} % Main chapter title

\label{Chapter5} % Change X to a consecutive number; for referencing this chapter elsewhere, use \ref{ChapterX}

%----------------------------------------------------------------------------------------
%	SECTION 1
%----------------------------------------------------------------------------------------
\section{Conclusiones generales}

La presente investigación apunta a la concepcion y realizacion de un sistema fotovoltaico, para lo cual se diseñó un \textit{harware} que funcionan como Generador Eléctrico Solar Autónomo (GESA) y un \textit{firmware} que permite controlar dispositivos para evitar que el problema se manifieste y si eventualmente se manifiesta, detectarlo e informarlo para tomar las acciones necesarias. En conjunto aportan una solución a un problema de disponibilidad de nodos debido a su autonomía energética.

Estos sistemas fotovoltáicos se presentan como una solución portable y de bajo costo que puede ser utilizada para alimentar dispositivos electronicos, como componentes programables, microcontroladores, FPGA, computadoras industriales y sistemas embebidos en general.

El presente trabajo es de interés en el área de Sistemas Embebidos para la implementación de otros algoritmos.

A partir del trabajo realizado se deduce que resulta necesario analizar el comportamiento del protocolo MAC basado en el tipo de tráfico que se tiene que manejar y un uso eficiente del radio, con la finalidad que dos nodos no se interfieran en sus transmisiones y buscando una solución para cuando suceda.

A su vez una buena práctica es analizar los avances de la investigación ejecutando la aplicación en casos prácticos.

Análisis de costos

\medskip
Durante el desarrollo de este trabajo integrador se aplicaron conocimientos adquiridos a lo largo de la Carrera de Especialización en Sistemas Embebidos. Principalmente de las siguientes asignaturas: 

\begin{itemize}
\item
\textbf{Arquitectura de microprocesadores}. Resultó necesario tener conocimientos sobre la Arquitectura ARM Cortex M para la programación de la plataforma \textit{Mote LSE} y su interacción con los periféricos.

\item
\textbf{Programación de microprocesadores}. Se utilizaron los principios de programación en Lenguaje C para microcontroladores y periféricos. Uso de repositorio. Diseño con el principio KISS\footnote{\textit{del inglés Keep It Simple, Stupid!: «¡Manténlo simple, estúpido!»}} que establece a la simplicidad como objetivo clave de diseño de cualquier sistema y cualquier complejidad innecesaria debe ser evitada. Mas aún cuando el ahorro de energía es un tema crítico para la aplicación y se desea evitar un mayor consumo en la operación. 

\item
\textbf{Gestión de Proyectos en Ingeniería}. El Plan de Proyecto Final, llevó a organizar el presente trabajo metódicamente. Se emplearon buenas practicas de gestión de proyectos que pueden llevarse a cabo en cualquier ámbito de la ingeniería, tanto laboral como académico.

\item 
\textbf{Protocolos de Comunicación}. Se aplicaron los conocimientos obtenidos sobre los protocolos de comunicación SPI, USB y 802.15.4.

\item 
\textbf{Introducción a Redes de Sensores Inalámbricos}. .
\end{itemize}

\medskip

\noindent Además, durante el desarrollo del trabajo final también se adquirieron conocimientos en:

\begin{itemize}
	\item Sistema de control de versiones de software.
	\item Red eléctrica inteligente (o REI; \textit{Smart Grid} en inglés).
	\item Energías alternativas.	
\end{itemize}

\medskip

Por lo tanto, se concluye que los objetivos planteados al comienzo del trabajo han sido alcanzados satisfactoriamente, cumpliendo con el requerimiento de mas alta prioridad: Presentar el Trabajo Final con la finalización del proyecto.

%----------------------------------------------------------------------------------------
%	SECTION 2
%----------------------------------------------------------------------------------------
\section{Trabajos futuros}

Hacer la adaptación necesaria del sistema para ser soportado por las plataformas disponibles del Proyecto CIAA \citep{CIAA} (Computadora Industrial Abierta Argentina)y de esta manera brindarle autonomía solar a aplicaciones industriales que no tengan disponibilidad de redes de distribución eléctrica .

Resulta esencial emplear protocolos de sincronización de tiempo, ya sea con corrección de reloj o con una tabla de traducción de tiempo, para lograr ciclos de trabajo eficientes y su desempeño es proporcional a la exactitud de la sincronización. Esto se puede lograr empleando protocolos de sincronización por parejas (por ejemplo TPSN y RBS) o a nivel de red (extensiones de TPSN y RBS).

desarrollar una implementacion donde convivan





