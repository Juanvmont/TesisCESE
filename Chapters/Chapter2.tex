\chapter{Introducción Específica} % Main chapter title

\label{Chapter2}

Este capítulo describe los objetivos del trabajo \ref{sec:trabajo}, las herramientas a utilizar \ref{sec:herramientas} y consideraciones de costo energético según el modo de operación y el tipo de trama \ref{sec:costo}. A continuación se muestran los requerimientos agrupados en \textit{hardware, software} y documentación y entregables\ref{sec:requerimientos} y un diagrama con la planificación del proyecto con un detalle de las tareas a realizar: AON y Diagrama de Gantt \ref{sec:tareas}.
%----------------------------------------------------------------------------------------
%	SECTION 1
%----------------------------------------------------------------------------------------

\section{Descripción del trabajo}
\label{sec:trabajo}
\noindent Los objetivos del trabajo son los siguientes:

\begin{itemize}
	\item Agregar al nodo \textit{Mote LSE} un circuito cargador a partir de un panel solar.
	\item Desarrollar un software de supervisión de nodos \textit{Mote LSE} con extensión de panel solar.
	\item Implementar el sistema en \textit{n} nodos desplegados en una red inalámbrica de área personal.
	\item Gestionar el modo de operación del nodo en función de la tensión que entrega el circuito cargador y la proyección de la batería restante.
	\item Supervisar la temperatura del entorno del nodo.
	\item Reportar a un nodo central el estado de salud del nodo.
	\item Fijar y leer en forma remota las alarmas/parámetros de configuración.
	\end{itemize}

%----------------------------------------------------------------------------------------
%	SECTION 2
%----------------------------------------------------------------------------------------
\section{Herramientas}
\label{sec:herramientas}

LPCxpresso es un toolchain que está compuesto por un IDE basado en eclipse, compiler y linker GNU, GDB debugger y también una herramienta que incluye un programador y debugger (LPC-Link) que es usado para programar el nodo \textit{Mote LSE}(Ver figura \ref{fig:mote}).

\vspace{10px}

\begin{figure}[h!]
	\centering
    \includegraphics[width=.8\textwidth]{./Figures/mote.jpg}
	\label{fig:mote}
	\caption{Fotografía de la plataforma: Nodo \textit{Mote LSE}}
\end{figure}

\noindent Esta plataforma se compone de:
\begin{itemize}
\item Microcontrolador NXP LPC1343
   \begin{itemize}
   \item Procesador Cortex-M3 @72MHz. 
   \item 32kB de memoria Flash.
   \item 8kB de memoria RAM.
   \end{itemize}
\item Transceptor TI-2520.
		\begin{itemize}
			\item 2394-2507 MHz.
			\item VCO, LNA, PA y filtros On-chip.
			\item 3 modos de potencia flexibles para consumo reducido.
			\item Muy bajo consumo de corriente.
			\begin{itemize}
				\item RX: 18.5 - 22.3 mA.
				\item TX: 25.8 - 33.6 mA.
			\end{itemize}
			\item Interfaz de usuario:
			\begin{itemize}
				\item SPI.
				\item 6 GPIOs.
 				\item Generador de interrupciones.
				\item Respuestas automáticas a diferentes eventos.
%				\item Modo de Packet Sniffer embebido
			\end{itemize}
		\end{itemize}
\item Extensor de rango TI-2591.
\item Antena y balun en microstrip.
\item Circuito de control de carga bq24080 1-A.
		\begin{itemize}
			\item Alimentación de 5V(DC) a través de conector USB micro-B.
			\item La carga de la batería termina basada en una corriente mínima.
			\item Entra en modo \textit{sleep} automáticamente cuando la entrada es removida.		
		\end{itemize}
\item Batería de Li-ion de 3.7V y 900mAh.
\item Sensor de luz.
\item Sensor de temperatura.
\end{itemize}

Instrumentos de medición. 

%----------------------------------------------------------------------------------------
%	SECTION 3
%----------------------------------------------------------------------------------------

\section{Costo energético}
\label{sec:costo}
consumos en cada modo de operación
comparación de tramas y throughput

%----------------------------------------------------------------------------------------
%	SECTION 4
%----------------------------------------------------------------------------------------

\clearpage
\section{Requerimientos}
\label{sec:requerimientos}

\noindent Requerimientos de hardware:
\begin{itemize}
	\item \textbf{REQ1:}\\ Compatibilidad con nodos Mote LSE.
	\item \textbf{REQ2:}\\ Agregar al nodo \textit{Mote LSE} la extensión de circuito cargador a partir de panel solar de 12V x 0.5A.
\end{itemize}
	
\noindent Requerimientos de software:
\begin{itemize}
	\item \textbf{REQ1:}\\ El sistema de monitoreo debe poder trabajar con una cantidad \textit{n} de nodos simultáneamente.
	\item \textbf{REQ2:}\\ El sistema debe soportar topologías de tipo estrella, peer-to-peer y árbol de cluster.
	\item \textbf{REQ3:}\\ El nodo tendrá una memoria no volátil donde guarda los valores de alarma, los parámetros monitoreados, el tiempo total de uso, y la cantidad de ciclos de carga/descarga.
	\item \textbf{REQ4:}\\ Todas las alarmas/parámetros de configuración/estado del nodo deben poder fijarse y leerse en forma remota (prioridad alta).	
	\item \textbf{REQ5:}\\ El cambio de modos entre batería y panel solar lo hace el nodo en función de la tensión que entrega el panel solar; el nodo debe funcionar en modo panel solar todo lo posible.
	\item \textbf{REQ6:}\\ La recarga de la batería la decide el nodo en función de la proyección de batería restante y la tensión que entrega el panel solar.	
	\item \textbf{REQ7:}\\ Los nodos deben monitorear en todo momento:	
	\begin{itemize}
		\item REQ7\_A: Consumo de corriente.
		\item REQ7\_B: Temperatura ambiente.
		\item REQ7\_C: Tensión que entrega el panel solar.
		\item REQ7\_D: Modo de funcionamiento (batería/panel solar).
		\item REQ7\_E: Alarmas cuando la temperatura ambiente es mayor o menor a determinados valores.
		\item REQ7\_F: Tiempo total que lleva el nodo funcionando.
		\end{itemize}
	\item \textbf{REQ8:}\\Los nodos en modo batería deben monitorear:
		\begin{itemize}
		\item REQ8\_A: Salud de la batería (tensión, corriente y temperatura).
		\item REQ8\_B: Proyección de batería restante (conociendo la batería y el consumo, estimar cuánto tiempo de vida le queda al nodo).
		\item REQ8\_C: Alarmas: Tiempo de vida proyectado del nodo menor a un nivel prefijado; tensión de batería menor a un valor prefijado, corriente de la batería mayor a un valor prefijado, y temperatura de batería mayor a un valor prefijado.
		\end{itemize}
	\item \textbf{REQ9:}\\Los nodos en modo panel solar deben monitorear:
		\item REQ9\_A: Estado del sistema de recarga (tensión y corriente que entrega el panel y tensión, corriente y temperatura de la batería).
		\item REQ9\_B: Tiempo restante estimado para recarga total de la batería.
		\item REQ9\_C: Cantidad de ciclos de carga/descarga acumulados.
		\item REQ9\_D: Si está recargando la batería.
	\item \textbf{REQ10:}\\Se debe implementar un algoritmo de ahorro de energía en los nodos en estado de alarma (prioridad baja).
\end{itemize}

\noindent Requerimientos de documentación y entregables:
\begin{itemize}
	\item \textbf{REQ1:}\\ El manual de uso debe especificar los parámetros de red configurables para implementar el sistema.
	\item \textbf{REQ2:}\\ Los diagramas de instalación debe contemplar las tres topologías posibles de acuerdo a 802.15.4.
	\item \textbf{REQ3:}\\ Se debe presentar un informe de avance del proyecto en una presentación pública ante el jurado el 31 de marzo de 2016.
	\item \textbf{REQ4:}\\ Se debe presentar un trabajo final con la finalización del proyecto. Este es el requerimiento de más alta prioridad.	
\end{itemize}

%----------------------------------------------------------------------------------------
%	SECTION 5
%----------------------------------------------------------------------------------------
\clearpage
\section{Tareas a realizar}
\label{sec:tareas}

La secuencia de tareas y sus duraciones se muestran en la figura \ref{fig:AON} 

\begin{figure}[h!]
	\centering
    \includegraphics[width=1\textwidth]{./Figures/AON.PNG}
	\label{fig:AON}
	\caption{Diagrama de \textit{Activity On Node}}
\end{figure}


\pagebreak[4]
\global\pdfpageattr\expandafter{\the\pdfpageattr/Rotate 90}
\includegraphics[width=1.1\paperwidth, angle = 90 ]{./Figures/GANTT.png}
\clearpage
\pagebreak[4]\global\pdfpageattr\expandafter{\the\pdfpageattr/Rotate 0}


%\begin{figure}[h!]
%	\centering
%    \includegraphics[width=1\textwidth]{./Figures/WBS1.PNG}
%	\label{fig:WBS1}
%	\caption{WBS parte 1}
%\end{figure}
%
%\begin{figure}[h!]
%	\centering
%    \includegraphics[width=0.7\textwidth]{./Figures/gantt1.PNG}
%	\label{fig:gantt1}
%	\caption{Diagrama de GANTT parte 1}
%\end{figure}
%
%\begin{figure}[h!]
%	\centering
%    \includegraphics[width=1\textwidth]{./Figures/WBS2.PNG}
%	\label{fig:WBS2}
%	\caption{WBS parte 2}
%\end{figure}
%
%\begin{figure}[h!]
%	\centering
%    \includegraphics[width=0.7\textwidth]{./Figures/gantt2.PNG}
%	\label{fig:gantt2}
%	\caption{Diagrama de GANTT parte 2}
%\end{figure}
%
%\begin{figure}[h!]
%	\centering
%    \includegraphics[width=1\textwidth]{./Figures/WBS3.PNG}
%	\label{fig:WBS3}
%	\caption{WBS parte 3}
%\end{figure}
%
%\begin{figure}[h!]
%	\centering
%    \includegraphics[width=0.7\textwidth]{./Figures/gantt3.PNG}
%	\label{fig:gantt3}
%	\caption{Diagrama de GANTT parte 3}
%\end{figure}
%\begin{itemize}
%	\item Planificación del proyecto (40 hs)
%	\item Tareas de hardware (55 hs):
%	\begin{itemize}
%	\item Análisis del hardware nodo Mote LSE y LPC-Link (20 hs).
%	\item Adaptación de circuito cargador a partir de panel solar (35 hs).
%	\end{itemize}
%	\item Tareas de software-topologías (40 hs):
%	\begin{itemize}
%	\item Organizar los nodos en topología de tipo estrella (10 hs).
%	\item Organizar los nodos en topología de tipo peer-to-peer (10 hs).
%	\item Organizar los nodos en topología de tipo árbol de cluster (20 hs).
%	\end{itemize}
%	\item Tareas de software-memoria (25 hs):
%	\begin{itemize}
%	\item Memoria no volátil (10 hs).
%	\item Escribir y leer en forma remota (15 hs).    
%	\end{itemize}
%	\item Tareas de software-energía (60 hs):
%	\begin{itemize}
%	\item Medir la tensión que entrega el panel solar (5 hs).
%	\item Cambio de modos (batería/panel) (10 hs).
%	\item Proyección de batería restante (20 hs).
%	\item Recarga de batería (25 hs).
%	\end{itemize}
%	\item Tareas de software-monitoreo (80 hs):
%	\begin{itemize}
%	\item Medir consumo de corriente (5 hs).
%	\item Modo de funcionamiento (20 hs).
%	\item Medir temperatura ambiente (15 hs).
%	\item Alarmas de temperatura (2) (20 hs).
%	\item Medir tiempo de funcionamiento (20 hs).
%	\end{itemize}
%	\item Tareas de software-batería (50 hs):
%	\begin{itemize}
%	\item Salud de batería (20 hs).
%	\item Generar alarmas (4) (30 hs).
%	\end{itemize}
%	\item Tareas de software-panel (70 hs):
%	\begin{itemize}
%	\item Estado del sistema de recarga (10 hs).
%	\item Medir tiempo restante estimado para recarga total de la batería (25 hs).
%	\item Cantidad de ciclos de carga/descarga acumulados (15 hs).
%	\item Estado de recarga (20 hs).
%	\end{itemize}
%	\item Tareas de software-ahorro (60 hs):
%	\begin{itemize}
%	\item Identificar los posibles valores de configuración para los parámetros de red (10 hs).
%	\item Definir el ciclo útil y la periodicidad de los reportes para el modo ahorro (5 hs).
%	\item Definir las funcionalidades reducidas para el modo ahorro (10 hs).
%	\item Desarrollar el algoritmo de ahorro de energía (15 hs).
%	\item Pruebas software-ahorro (20 hs).
%	\end{itemize}
%	\item Tareas de software-integración (20 hs):
%	\begin{itemize}
%	\item Integrar todas las funcionalidades de software (5 hs).
%	\item Pruebas software-integración (15 hs).  
%	\end{itemize}
%	\item Tareas de documentación y entregables (80 hs):
%	\begin{itemize}
%	\item Documentación de los diagramas de instalación (10 hs).
%	\item Documentación del informe de avance. (10 hs).
%	\item Documentación del manual de uso (20 hs).
%	\item Documentación del informe final (40 hs).
%	\end{itemize}
%	\item Procesos de cierre (20 hs).
%	\begin{itemize}
%	\item Seguimiento del Gantt (4 hs)
%	\item Organización de la presentación pública(3 hs)    
%	\item Documentación de lecciones aprendidas (4 hs)
%	\item Documentación de problemas y soluciones (4 hs)
%	\item Comparación de horas estimadas y empleadas (2 hs)
%	\item Organización de acto de agradecimiento (3 hs)
%	\end{itemize}
%\end{itemize}