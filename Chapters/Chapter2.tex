% Chapter Template

\chapter{Introducción Específica} % Main chapter title

\label{Chapter2} % Change X to a consecutive number; for referencing this chapter elsewhere, use \ref{ChapterX}

%----------------------------------------------------------------------------------------
%	SECTION 1
%----------------------------------------------------------------------------------------

\section{Trabajo a realizar}


%-----------------------------------
%	SUBSECTION 1
%-----------------------------------
%\subsection{Objetivo}
\begin{itemize}

	\item \textbf{Objetivos}:\\
	Desarrollar un software que permita utilizar la CIAA como un controlador de acuarios. Agregar valor al ecosistema del proyecto-ciaa incorporando nuevas aplicaciones para la misma.
	
	\item \textbf{Alcance}:\\
	El proyecto incluye únicamente el desarrollo de un software de control para la plataforma CIAA.

%	\item \textbf{Restricciones}:\\
%	Finalización del proyecto 15 de diciembre de 2015

	\item \textbf{Suposiciones}:\\
	Contar con los fondos solicitados al Ministerio de Industria Nacional permitirá costear los sensores y actuadores para armar una planta piloto de desarrollo.  En caso que los fondos no estén disponibles tanto por el rechazo del proyecto presentado como por una demora en los plazos de adjudicación, está contemplado la utilización de simulaciones para emular los elementos que no se hayan podido adquirir.
	
	Asimismo, se estima que estará disponible para el desarrollo del proyecto, y mientras dure el mismo, una plataforma CIAA-NXP.
	
	\end{itemize}
%-----------------------------------
%	SUBSECTION 2
%-----------------------------------

%\subsection{Alcances}
%Morbi rutrum odio eget arcu adipiscing sodales. Aenean et purus a est pulvinar pellentesque. Cras in elit neque, quis varius elit. Phasellus fringilla, nibh eu tempus venenatis, dolor elit posuere quam, quis adipiscing urna leo nec orci. Sed nec nulla auctor odio aliquet consequat. Ut nec nulla in ante ullamcorper aliquam at sed dolor. Phasellus fermentum magna in augue gravida cursus. Cras sed pretium lorem. Pellentesque eget ornare odio. Proin accumsan, massa viverra cursus pharetra, ipsum nisi lobortis velit, a malesuada dolor lorem eu neque.

%-----------------------------------
%	SUBSECTION 3
%-----------------------------------
\clearpage
\subsection{Requerimientos y Criterios de aceptación}

\begin{itemize}
	\item \textbf{REQ1:}\\ El \textit{software} se debe desarrollar para la plataforma de hardware CIAA, en lenguaje C.
	\item \textbf{REQ2:}\\ El equipo a desarrollar debe poder medir las siguientes variable del entorno de un acuario,
	\begin{itemize}
		\item REQ2\_A: temperatura en el rango de 16 a 30 %°C con una resolución de 0,5 °C o superior.
		\item REQ2\_B: pH en el rango de 5 a 9 con una resolución de 0,1 o superior.
%		\item REQ2\_C: nivel de agua alto o bajo
	\end{itemize}
	\item \textbf{REQ3:}\\ El equipo debe poder actuar sobre los siguiente elementos del entorno de un acuario,
	\begin{itemize}
		\item REQ3\_A: Encender/apagar la iluminación
		\item REQ3\_B: Encender/apagar bombas para el llenado/vaciado de agua
		\item REQ3\_C: Encender/apagar bomba de CO2
%		\item REQ3\_C: Encender/apagar dosificador de alimento/nutrientes
		\item REQ3\_D: Encender/apagar calefactor
	\end{itemize}
	\item \textbf{REQ4:}\\ El equipo debe poder ser administrado mediante una interfaz web. Se debe poder visualizar el estado del sistema y realizar su programación.
	\item \textbf{REQ5:}\\ El equipo debe poder emitir una alarma por un medio adecuado, sonoro y/o visual cuando las variables bajo control salgan de su rango de operación segura.
\end{itemize}

\textbf{Criterio de aceptación}:
	\begin{itemize}
		\item Correcta visualización de todas las variable de estado del sistema enumeradas en el REQ2 mediante una interfaz web.
		\item Correcto comportamiento al accionar los actuadores enumerados en el REQ3 desde la interfaz web.
	\end{itemize}

%-----------------------------------
%	SUBSECTION 4
%-----------------------------------

\subsection{Desglose de tareas / GANTT}
Morbi rutrum odio eget arcu adipiscing sodales. Aenean et purus a est pulvinar pellentesque. Cras in elit neque, quis varius elit. Phasellus fringilla, nibh eu tempus venenatis, dolor elit posuere quam, quis adipiscing urna leo nec orci. Sed nec nulla auctor odio aliquet consequat. Ut nec nulla in ante ullamcorper aliquam at sed dolor. Phasellus fermentum magna in augue gravida cursus. Cras sed pretium lorem. Pellentesque eget ornare odio. Proin accumsan, massa viverra cursus pharetra, ipsum nisi lobortis velit, a malesuada dolor lorem eu neque.

%----------------------------------------------------------------------------------------
%	SECTION 2
%----------------------------------------------------------------------------------------

\section{Planteo del problema}

Sed ullamcorper quam eu nisl interdum at interdum enim egestas. Aliquam placerat justo sed lectus lobortis ut porta nisl porttitor. Vestibulum mi dolor, lacinia molestie gravida at, tempus vitae ligula. Donec eget quam sapien, in viverra eros. Donec pellentesque justo a massa fringilla non vestibulum metus vestibulum. Vestibulum in orci quis felis tempor lacinia. Vivamus ornare ultrices facilisis. Ut hendrerit volutpat vulputate. Morbi condimentum venenatis augue, id porta ipsum vulputate in. Curabitur luctus tempus justo. Vestibulum risus lectus, adipiscing nec condimentum quis, condimentum nec nisl. Aliquam dictum sagittis velit sed iaculis. Morbi tristique augue sit amet nulla pulvinar id facilisis ligula mollis. Nam elit libero, tincidunt ut aliquam at, molestie in quam. Aenean rhoncus vehicula hendrerit.


%-----------------------------------
%	SUBSECTION 1
%-----------------------------------
\subsection{Descripción del problema real}

Interacciones entre variables de estado.

Suposiciones, limitaciones y modelo adoptado.

%-----------------------------------
%	SUBSECTION 2
%-----------------------------------

\subsection{¿Qué hace falta medir, actuar, alertar?}

en referencia a los requerimientos y al recorte del problema de la sección anterior

%-----------------------------------
%	SUBSECTION 3
%-----------------------------------

\subsection{Alternativas tecnológicas}

Alternativas de sensores.
%\begin{figure}[h!]
%	\centering
%    \includegraphics[width=.5\textwidth]{./Figures/sensor_nivel}
%	\label{fig:Sensor de nivel}
%	\caption{Sensor de nivel de Agua}
%\end{figure}
%
%\begin{figure}[h!]
%	\centering
%    \includegraphics[width=.5\textwidth]{./Figures/sensor_temp}
%	\label{fig:Sensor de temperatura}
%	\caption{Sensor de temperatra}
%\end{figure}
%Alternativas de actuadores.
%
%\begin{figure}[h!]
%	\centering
%    \includegraphics[width=.5\textwidth]{./Figures/actuador_heater}
%	\label{fig:act_heater}
%	\caption{Actuador Calefactor}
%\end{figure}
%
%\begin{figure}[h!]
%	\centering
%    \includegraphics[width=.5\textwidth]{./Figures/actuador_pump}
%	\label{fig:act_pump}
%	\caption{Actuador Bomba de agua}
%\end{figure}

Alternativas de alarmas.

listar y justificar alternativas elegidas.