% Chapter 1

\chapter{Introducción General} % Main chapter title

\label{Chapter1} % For referencing the chapter elsewhere, use \ref{Chapter1} 
\label{IntroGeneral}

%----------------------------------------------------------------------------------------

% Define some commands to keep the formatting separated from the content 
\newcommand{\keyword}[1]{\textbf{#1}}
\newcommand{\tabhead}[1]{\textbf{#1}}
\newcommand{\code}[1]{\texttt{#1}}
\newcommand{\file}[1]{\texttt{\bfseries#1}}
\newcommand{\option}[1]{\texttt{\itshape#1}}
\newcommand{\grados}{$^{\circ}$}

%----------------------------------------------------------------------------------------

%\section{Introducción}
El nodo \textit{Mote LSE}. Sus características principales se tratan en la sección \ref{sec:herramientas}.


\section{Motivación}
\label{sec:Motivación}

La justificación de este proyecto es proponer una solución preventiva a la falta de disponibilidad debido a fallas de energía o a temperatura de entorno de nodos \textit{Mote LSE} con sistema autónomo de batería, que reporte alarmas de salud a un nodo central y tome acciones para el ahorro de energía.

La necesidad surge como complemento en el despliegue de nodos que soportan el estándar IEEE 802.15.4 \citep{802.15.4} para \textit{Low-Rate Wireless Personal Area Networks (LR-WPANs)} de la \textit{IEEE Standards Association}.

El cual define su operación basado en un modelo simple de capas:
		\begin{itemize}
			\item Nivel físico (PHY)
			\item Control de acceso al medio (MAC)
		\end{itemize}

\noindent Sus principales características son:
		\begin{itemize}
			\item Uso doméstico e industrial.
			\item Comunicaciones simples de bajo costo. 
			\item Bajas tasas de transferencia (throughput).
			\item Extremadamente bajo consumo de potencia.
			\item Confiabilidad en la transferencia de datos.
			\item Opera en una banda de frecuencia sin licencia.
		\end{itemize}

La intensión de este estándar fue establecer los lineamientos para aplicaciones con limitaciones de potencia, como lo son los dispositivos con fuente de alimentación autónoma, como por ejemplo baterías, paneles solares y otras fuentes de energía alternativas. Para lograrlo, hace énfasis en el ahorro de energía gracias a la simplicidad de su estructura principalmente, proponiendo un modelo de muy pocas capas (dos: PHY y MAC) y al controlar el ciclo de trabajo.

El tiempo de transmisión como una proporción del intervalo de tiempo entre transmisiones, es llamado ciclo de trabajo. El uso de la batería es optimizado con 802.15.4 al usar ciclos de trabajo extremadamente bajos. 
		
Con la extensión del circuito cargador se aumenta la autonomía del nodo \textit{Mote LSE} y con el \textit{firmware} desarrollado el nodo será capaz de determinar el modo de operación entre batería y panel solar en función de la tensión que entrega el circuito cargador del panel y la proyección de batería restante.

El trabajo propuesto es de interés en el área de Sistemas Embebidos para implementar conjuntamente con otro algoritmos en redes WSN.

\section{¿Qué es WSN?}
\label{sec:wsn}

WSN (por sus siglas en inglés \textit{Wireless Sensor Networks}), o Redes de Sensores Inalámbricos, no es mas que una serie de dispositivos sensores para monitorear fenómenos físicos, desplegados en una red y conectados inalámbricamente.

\noindent Las áreas de aplicación pueden ser:
		\begin{itemize}
			\item Medición inteligente.
	%	\item Comunicaciones y Control de ferrocarril
	%	\item Redes de monitoreo de infraestructuras criticas
			\item Domótica y seguridad.
			\item Productos electrónicos de consumo.
			\item Cuidado de la salud.
			\item Control y monitoreo de vehículos.
			\item Agricultura.
		\end{itemize}
		
Estas redes pueden ir desde Wireless PANs a Wireless Regional Area Networks y una vez centralizada gateway
%In computing, a sink, event sink or data sink is a class or function designed to receive incoming events from another object or function. This is commonly implemented in C++ as callbacks. Object-oriented languages, such as Java and C#, have built-in support for sinks by allowing events to be fired to delegate functions.

%Due to lack of formal definition, a sink is often miscontrued with a gateway which is a similar construct but the latter is usually either an end-point or allows bi-direction communication between dissimilar systems, as opposed to just an event input point[citation needed]. This is often seen in C++ and hardware-related programming[citation needed], thus the choice of nomenclature by a developer usually depends on whether the agent acting on a sink is a producer or consumer of the sink content.

La tabla \ref{tab:802} lista los estándares del proyecto IEEE 802 o LMSC (LAN/MAN Standards Committee).

\vspace{10px}

\begin{table}[ht]
	\centering
	\caption{Estandares 802®}
	\begin{tabular}{@{} l *1c @{}}    \toprule
		\emph{\textbf{Estandares 802®}} \\
		%& \emph{\textbf{Valor}} & \emph{\textbf{Unidad}
		\midrule
%		Potencia nominal	& 10 	& Wp	\\	
%		Tensión a PN		& 17.4	& V\\
%		Corriente a PN	& 0.58		& A\\
%		Dimensiones		& 301x352x22 	& mm\\
%		Peso				& 0.58		& Kg	\\
        802: Overview & Architecture\\
        802.1: Bridging & Management\\
        802.2: Logical Link Control\\
        802.3: Ethernet\\
        802.11: Wireless LANs\\
        802.15: Wireless PANs\\
        802.16: Broadband Wireless MANs\\
        802.17: Resilient Packet Rings\\
        802.19: TV White Space Coexistence Methods\\
        802.20: Mobile Broadband Wireless Access\\
        802.21: Media Independent Handover Services\\
        802.22: Wireless Regional Area Networks\\
		\bottomrule
		\hline
	\end{tabular}
	\label{tab:802}
\end{table}


\section{Implicaciones de energía}
\label{sec:energía}

Soluciones existentes y tecnología de baterías. Estado del arte de paneles y baterías.

El módulo fotovoltáico, mas conocido como panel solar, se encarga de convertir la energía solar en electricidad a partir de la luz que incide sobre él, mediante el efecto fotoeléctrico. Junto con otros pocos componentes, como el banco de baterías para acumular la energía generada por el (los) módulo (s) y una serie de elementos auxiliares tales como reguladores de carga de las baterías, reguladores de voltaje, inversores de corriente y otros según el tipo de aplicación y el consumo al que estén destinados, funcionan como Generadores Eléctricos Solares Autónomos (GESA).

Son comúnmente usados para abastecer medianos a bajos consumos de energía eléctrica, donde no hay red de distribución eléctrica o el acceso al mismo es reducido. Para lograrlo es necesario dar al generador solar la posición adecuada: orientación norte, inclinación de acuerdo al lugar geográfico y en lo posible, libre de sombras la mayor parte del tiempo.

Los avances en la investigación en el área han logrado disminuir el precio de las células de silicio cristalino. La figura \ref{fig:swanson} muestra la evolución del precio entre 1977 y 2015 \footnote{fuente: Bloomberg New Energy Finance. De Hanjin - Trabajo propio, CC BY-SA 3.0, https://commons.wikimedia.org/w/index.php?curid=29498744}.


\begin{figure}[h!]
	\centering
    \includegraphics[width=0.7\textwidth]{./Figures/SwansonEffect.png}
	\label{fig:swanson}
	\caption{Evolución del precio de células fotovoltaicas de silicio cristalino (USD/Wp)}
\end{figure}
%----------------------------------------------------------------------------------------
