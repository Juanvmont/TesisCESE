% Chapter 1

\chapter{Introducción General} % Main chapter title

\label{Chapter1} % For referencing the chapter elsewhere, use \ref{Chapter1} 
\label{IntroGeneral}

%----------------------------------------------------------------------------------------

% Define some commands to keep the formatting separated from the content 
\newcommand{\keyword}[1]{\textbf{#1}}
\newcommand{\tabhead}[1]{\textbf{#1}}
\newcommand{\code}[1]{\texttt{#1}}
\newcommand{\file}[1]{\texttt{\bfseries#1}}
\newcommand{\option}[1]{\texttt{\itshape#1}}

%----------------------------------------------------------------------------------------

%\section{Introducción}

En la presente memoria se documenta el trabajo final de la Carrera de Especialización en Sistemas Embebidos que consiste en implementar un \textit{web server} utilizando la plataforma CIAA-NXP, el RTOS freeRTOS V8.0.1, y el stack TCP/IP lwip V1.41. 

Se busca satisfacer una necesidad relacionada con los acuarios. En particular, interesa poder controlar un número de variables indicativas del estado del mismo.  Se busca implementar una interfaz web que permita visualizar información relevante sobre el estado del acuario y permita actuar sobre las variables cuyas desviaciones respecto a valores de referencia, pueden afectar negativamente al entorno.  Asimismo, se busca facilitar tareas de mantenimiento que se deben realizar en forma rutinaria y que son de vital importancia para la supervivencia de los seres vivos que habitan en el acuario, principalmente peces, invertebrados y plantas. Detalles referentes a qué son los acuarios serán abordados en la sección \ref{sec:acuario}.

Se implementa un firmware de monitoreo y control sobre una de las plataformas de hardware desarrollada por el Proyecto Computadora Industrial Abierta Argentina, denominado de ahora en adelante Proyecto-CIAA, en particular la versión CIAA-NXP que utiliza un microprocesador fabricado por la empresa NXP. El Proyecto-CIAA se trata en la sección \ref{sec:proyecto-ciaa}.

Debido a la vasta difusión de las redes de computadoras y la tendencia mundial a incorporar cada día más dispositivos a internet, fenómeno también conocido con el internet de las cosas o IoT (por sus siglas en inglés \textit{Internet of Things}) se opta por desarrollar una interfaz web embebida en la CIAA-NXP.  Dicha interfaz es accesible mediante el protocolo ethernet a través de una red de área local o LAN (por sus siglas en inglés \textit{Local Area Network}). Detalles de la implementación se abordan en la sección \ref{sec:interfaz-web}.


\section{¿Qué es el Proyecto CIAA?}
\label{sec:proyecto-ciaa}

El Proyecto CIAA nació en 2013 como una iniciativa conjunta entre el sector académico y el industrial, representados por la ACSE\footnote{\url{http://www.sase.com.ar/asociacion-civil-sistemas-embebidos}} y CADIEEL\footnote{\url{http://www.cadieel.org.ar}}, respectivamente.

Objetivos del proyecto:

\begin{itemize}
	\item Impulsar el desarrollo tecnológico nacional, a partir de sumar valor agregado al trabajo y a los productos y servicios, mediante el uso de sistemas electrónicos, en el marco de la vinculación de las instituciones educativas y el sistema científico-tecnológico con la industria.
	\item Darle visibilidad positiva a la electrónica argentina.
	\item Generar cambios estructurales en la forma en la que se desarrollan y utilizan en nuestro país los conocimientos en el ámbito de la electrónica y de las instituciones y empresas que hacen uso de ella.
\end{itemize}

Todo esto en el marco de un trabajo libre, colaborativo y articulado entre industria y academia. 


\section{¿Qué es un Acuario?}
\label{sec:acuario}

El Acuarismo es una actividad comercial y recreativa ampliamente difundida en la Argentina que consiste en establecer y mantener un ecosistema acuático artificial. El término acuario o pecera se utiliza indistintamente para referirse tanto al contenedor de agua como a todo el ecosistema artificial. 

Un acuario puede albergar a un número determinado de peces, plantas e invertebrados y para su mantenimiento es preciso recrear un biotopo\footnote{Territorio o espacio vital cuyas condiciones ambientales son las adecuadas para que en él se desarrolle una determinada comunidad de seres vivos.} en función de los hábitos y costumbres de las especies que lo habiten; a modo de ejemplo, existen vertebrados acuáticos aptos para vivir en agua fría, otros requieren un sistema calefactor para mantener el agua en un microclima cálido, incluso las especies marinas precisan cuidados más específicos en cuanto al tratamiento del agua.

Los acuarios se clasifican principalmente según las condiciones del agua:
\begin{itemize}
	\item Por temperatura
	\begin{itemize}
		\item Agua fría; entorno a los 18 \degree C.
		\item Agua tropical; normalmente entre 22 y 27 \degree C.
	\end{itemize}
	\vspace{5px}
	\item Por salinidad
	\begin{itemize}
		\item Agua dulce; baja concentración de sales disueltas.
		\item Salobre; agua con entre 0.5 y 30 gr. de sal por litro.
		\item Agua de mar o Marinos; 35 gr. de sal por litro. 
	\end{itemize}
\end{itemize}

El punto de entrada recomendado para principiantes en la disciplina es el acuario de agua dulce y fría de al menos 90 litros.  Su mantenimiento es más sencillo que el de los acuarios marinos y los peces de agua fría son los más resistentes y los que presentan mayor compatibilidad entre especies. Normalmente se pueden mantener con agua de red debidamente declorada y acondicionada. Finalmente, un acuario más grande es más fácil de mantener, sobre todo en lo referido al tratamiento del agua y a conseguir una temperatura estable para los peces \citep{paradais1}.

Es necesario que el agua se encuentre debidamente tratada de acuerdo con las necesidades específicas que requiera la especie que se pretenda albergar. Según la salinidad del agua es necesario tener en cuenta los siguientes parámetros.

\begin{itemize}
\item Agua dulce:
	\begin{itemize}
		\item dureza del agua (dGH).
		\item valor pH.
		\item el CO2  o anhídrido carbónico.
	\end{itemize}
	\vspace{5px}
	\item Agua Salada:
	\begin{itemize}
		\item desidad del agua
		\item valor pH.
		\item el CO2  o anhídrido carbónico.
	\end{itemize}
	
\end{itemize}
Nos referimos a la dureza del agua (dGH), el valor pH y el CO2  o anhídrido carbónico. Existen en el mercado productos específicos que ayudan a mantenerla en óptimas condiciones, incluso \textit{tests} que facilitan un control adecuado sobre la medición de los valores químicos que debe tener el agua del acuario.

A modo de ejemplo, destacar que para acuarios de agua dulce, la media ideal del agua, tendrá como valores: un pH de 7, dureza 10º a 15º dGH y 5 a 10º KH, pero estas medidas son variables en función de la especie en cuestión.

Los acuarios de agua salada requieren mayor complejidad, entra en juego la salinidad del agua que en cierta medida sustituye a la dureza, y su valor debe estar en una escala de entre 1022-1025, por otro lado el pH mantendrá un valor a partir de 8, normalmente la media oscila en torno a 8,2.

%
%\begin{figure}[]
%	\centering
%    \includegraphics[width=.5\textwidth]{./Figures/acuarioHobby3.jpg}
%	\label{fig:acuarioHobby}
%	\caption{Acuario de un individuo particular}
%\end{figure}       
%
%\begin{figure}[]
%	\centering
%    \includegraphics[width=.7\textwidth]{./Figures/acuarioTienda.jpg}
%	\label{fig:acuarioTienda}
%	\caption{Tienda de insumos para Acuario}
%\end{figure}


%----------------------------------------------------------------------------------------

\section{Motivación}

Según el tipo de actividad se pueden encontrar acuarios con fines productivos para la recreación de especies destinadas al consumo humano.  Por otra parte

De conversaciones con personas que practican el acuarismo se tomó conocimiento de la existencia de una necesidad no satisfecha de un equipo electrónico que permita controlar las variables de estado de un acuario. 
 
Las personas consultadas manifestaron no conocer equipos que puedan realizar estas u otras funciones más complejas y que se comercialicen en el mercado local.  

Asimismo, se consultaron distintas locales de venta de insumos para acuarios de capital federal vía e-mail preguntando si tenían a la venta o conocían la existencia de un equipo para satisfacer la necesidad planteada y la respuesta fue abrumadoramente negativa. Como se detallará en la sección \ref{sec:existentes}, existen soluciones comerciales disponibles pero no se comercializan en el mercado local, lo cuál permite afirmar que la necesidad planteada no esta debidamente satisfecha y existe un mercado potencial que se analizará en la sección \ref{sec:mercado}.
%
%\begin{figure}[h!]
%	\centering
%    \includegraphics[width=.9\textwidth]{./Figures/nemo.jpg}
%	\label{fig:nemo}
%	\caption{Pez payaso}
%\end{figure}

\subsection{Dimensionamiento del mercado}
\label{sec:mercado}

\subsection{Soluciones existentes}
\label{sec:existentes}

%\begin{minipage}[c]{1.0\linewidth}
%\begin{minipage}[c]{0.6\linewidth}
%	\begin{itemize}
%		\item Ecosistema vivo y dinámico 
%		% Recreación de un ambiente subacuático para albergar peces, invertebrados y plantas
%		\vspace{10px}
%		\item Interacciones complejas
%		\vspace{10px}
%		\item Uso recreativo o comercial
%		\vspace{10px}
%		\item Malas condiciones = \$
%		\vspace{10px}
%  	\end{itemize}	
%  \end{minipage}
%  \begin{minipage}[c]{0.35\linewidth}
%	\begin{figure}[H]
%		{\includegraphics[width=1\textwidth]{./imagenes/acuario.jpg}}
%	\end{figure}	  	  	
%  \end{minipage}
%\end{minipage}
%\end{frame}

%\begin{figure}[h!]
%	\centering
%    \includegraphics[width=.5\textwidth]{./Figures/profilux}
%	\label{fig:competencia1}
%	\caption{Solución existente 1}
%\end{figure}
%
%\begin{figure}[h!]
%	\centering
%    \includegraphics[width=.5\textwidth]{./Figures/reefkeeper}
%	\label{fig:competencia2}
%	\caption{Solución existente 2}
%%	\footnote{Referencia a la figura}
%\end{figure}


\subsection{Aplicabilidad a otras áreas}





%----------------------------------------------------------------------------------------
