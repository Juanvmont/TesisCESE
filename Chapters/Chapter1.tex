% Chapter 1

\chapter{Introducción General} % Main chapter title

\label{Chapter1} % For referencing the chapter elsewhere, use \ref{Chapter1} 
\label{IntroGeneral}

%----------------------------------------------------------------------------------------

% Define some commands to keep the formatting separated from the content 
\newcommand{\keyword}[1]{\textbf{#1}}
\newcommand{\tabhead}[1]{\textbf{#1}}
\newcommand{\code}[1]{\texttt{#1}}
\newcommand{\file}[1]{\texttt{\bfseries#1}}
\newcommand{\option}[1]{\texttt{\itshape#1}}

%----------------------------------------------------------------------------------------

\section{Introducción}

En el trabajo que se documenta en la presente memoria, se busca satisfacer una necesidad relacionada con los \textbf{\texttt{acuarios}}. En particular, interesa poder monitorear y controlar un número de variables indicativas del estado del mismo.  Se busca obtener una plataforma que permita visualizar información relevante sobre el estado del acuario y permita actuar sobre las variables cuyas desviaciones respecto a valores de referencia, pueden afectar negativamente al entorno.  Asimismo, se busca facilitar tareas de mantenimiento que se deben realizar rutinariamente y que son de vital importancia para la supervivencia de los seres vivos que habitan en el acuario, principalmente peces, invertebrados y plantas. Detalles referentes a qué son los acuarios serán abordados en la sección \ref{sec:acuario}.

Se implementa un \texttt{firmware} de monitoreo y control sobre una de las plataformas de \texttt{hardware} desarrollada por el Proyecto \textit{Computadora Industrial Abierta Argentina}, denominado de ahora en adelante \textbf{\texttt{Proyecto-CIAA}}, en particular la versión \textbf{\texttt{CIAA-NXP}} que utiliza un microprocesador fabricado por la empresa NXP. El Proyecto-CIAA se trata en la sección \ref{sec:proyecto-ciaa}.

Debido a la vasta difusión de las redes de computadoras y la tendencia mundial a incorporar cada día más dispositivos a internet, fenómeno también conocido con el internet de las cosas o IoT (por sus siglas en inglés \textit{Internet of Things}) se opta por desarrollar una \textbf{\texttt{interfaz web embebida}} en la \texttt{CIAA-NXP}.  Dicha interfaz es accesible mediante el protocolo \texttt{ethernet} a través de una red de área local o LAN (por sus siglas en inglés \textit{Local Area Network}). Detalles de la implementación se abordan en la sección \ref{sec:interfaz-web}.




\subsection{¿Qué es un Acuario?}
\label{sec:acuario}

El Acuarismo es una actividad comercial y recreativa ampliamente difundida en la Argentina que consiste en establecer y mantener un ecosistema acuático artificial. La clasificación más importante de un acuario es respecto a la salinidad del agua contenida en él:

\begin{itemize}
\item Agua dulce: Baja concentración de sales disueltas.
\item Salobre: Agua con entre 0.5 y 30 gramos de sal por litro.
\item Agua de mar: 35 gr sal por litro% 
\end{itemize}

Sus orígenes históricos se remontan a la llegada de los primeros inmigrante europeos, principalmente alemanes que introdujeron las primeras especies de peces, \textit{Scalares}, \textit{Lebistes} y \textit{Bettas} entre otros.

El acuarismo en la Argentina antes de 1950 era materia para exclusivos y muy fanáticos. La mayoría de las especies resultaban muy onerosas y prácticamente sólo los \textit{Carassius} tenían precios accesibles para la mayoría de los potenciales interesados en esta actividad.

Desde la década del 70 a esta parte, se puede apreciar una importante difusión del acuarismo. Surgen nuevas ideas y nuevas publicaciones.  Actualmente, existen numerosas asociaciones de aficionados y publicaciones especializadas, algunas con tiradas de 100.000 ejemplares.  Entre las asociaciones más destacadas podemos nombrar a la Asociación Acuariófila Argentina y El Acuarista, entre otras.


\begin{figure}[]
	\centering
    \includegraphics[width=.5\textwidth]{./Figures/acuarioHobby3.jpg}
	\label{fig:acuarioHobby}
	\caption{Acuario de un individuo particular}
\end{figure}       

\begin{figure}[]
	\centering
    \includegraphics[width=.7\textwidth]{./Figures/acuarioTienda.jpg}
	\label{fig:acuarioTienda}
	\caption{Tienda de insumos para Acuario}
\end{figure}

\subsection{¿Qué es el Proyecto CIAA?}
\label{sec:proyecto-ciaa}

El Proyecto CIAA nació en 2013 como una iniciativa conjunta entre el sector académico y el industrial, representados por la ACSE\footnote{\url{http://www.sase.com.ar/asociacion-civil-sistemas-embebidos}} y CADIEEL\footnote{\url{http://www.cadieel.org.ar}}, respectivamente.

Objetivos del proyecto:

\begin{itemize}
	\item Impulsar el desarrollo tecnológico nacional, a partir de sumar valor agregado al trabajo y a los productos y servicios, mediante el uso de sistemas electrónicos, en el marco de la vinculación de las instituciones educativas y el sistema científico-tecnológico con la industria.
	\item Darle visibilidad positiva a la electrónica argentina.
	\item Generar cambios estructurales en la forma en la que se desarrollan y utilizan en nuestro país los conocimientos en el ámbito de la electrónica y de las instituciones y empresas que hacen uso de ella.
\end{itemize}

Todo esto en el marco de un trabajo libre, colaborativo y articulado entre industria y academia. 

%----------------------------------------------------------------------------------------

\section{Motivación}

De conversaciones con personas que practican el acuarismo se tomó conocimiento de la existencia de \textbf{\texttt{una necesidad no satisfecha}} de un equipo electrónico que permita monitoriar y controlar las variables de estado de un acuario. 
 
Las personas consultadas manifestaron no conocer equipos que puedan realizar estas u otras funciones más complejas y que se comercialicen en el mercado local.  

Asimismo, se consultaron distintas locales de venta de insumos para acuarios de capital federal vía e-mail preguntando si tenían a la venta o conocían la existencia de un equipo para satisfacer la necesidad planteada y la respuesta fue abrumadoramente negativa. Como se detallará en la sección \ref{sec:existentes}, existen soluciones comerciales disponibles pero no se comercializan en el mercado local, lo cuál permite afirmar que la necesidad planteada no esta debidamente satisfecha y existe un mercado potencial que se analizará en la sección \ref{sec:mercado}.

\begin{figure}[h!]
	\centering
    \includegraphics[width=.9\textwidth]{./Figures/nemo.jpg}
	\label{fig:nemo}
	\caption{Pez payaso}
\end{figure}

\subsection{Dimensionamiento del mercado}
\label{sec:mercado}

\subsection{Soluciones existentes}
\label{sec:existentes}

%\begin{minipage}[c]{1.0\linewidth}
%\begin{minipage}[c]{0.6\linewidth}
%	\begin{itemize}
%		\item Ecosistema vivo y dinámico 
%		% Recreación de un ambiente subacuático para albergar peces, invertebrados y plantas
%		\vspace{10px}
%		\item Interacciones complejas
%		\vspace{10px}
%		\item Uso recreativo o comercial
%		\vspace{10px}
%		\item Malas condiciones = \$
%		\vspace{10px}
%  	\end{itemize}	
%  \end{minipage}
%  \begin{minipage}[c]{0.35\linewidth}
%	\begin{figure}[H]
%		{\includegraphics[width=1\textwidth]{./imagenes/acuario.jpg}}
%	\end{figure}	  	  	
%  \end{minipage}
%\end{minipage}
%\end{frame}

\begin{figure}[h!]
	\centering
    \includegraphics[width=.5\textwidth]{./Figures/profilux}
	\label{fig:competencia1}
	\caption{Solución existente 1}
\end{figure}

\begin{figure}[h!]
	\centering
    \includegraphics[width=.5\textwidth]{./Figures/reefkeeper}
	\label{fig:competencia2}
	\caption{Solución existente 2}
%	\footnote{Referencia a la figura}
\end{figure}


\subsection{Aplicabilidad a otras áreas}





%----------------------------------------------------------------------------------------
