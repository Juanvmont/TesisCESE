% Chapter 1

\chapter{Introducción General} % Main chapter title

\label{Chapter1} % For referencing the chapter elsewhere, use \ref{Chapter1} 
\label{IntroGeneral}

%----------------------------------------------------------------------------------------

% Define some commands to keep the formatting separated from the content 
\newcommand{\keyword}[1]{\textbf{#1}}
\newcommand{\tabhead}[1]{\textbf{#1}}
\newcommand{\code}[1]{\texttt{#1}}
\newcommand{\file}[1]{\texttt{\bfseries#1}}
\newcommand{\option}[1]{\texttt{\itshape#1}}

%----------------------------------------------------------------------------------------

\section{Introducción}

\LaTeX{} is not a \textsc{wysiwyg} (What You See is What You Get) program, unlike word processors such as Microsoft Word or Apple's Pages. Instead, a document written for \LaTeX{} is actually a simple, plain text file that contains \emph{no formatting}. You tell \LaTeX{} how you want the formatting in the finished document by writing in simple commands amongst the text, for example, if I want to use \emph{italic text for emphasis}, I write the \verb|\emph{text}| command and put the text I want in italics in between the curly braces. This means that \LaTeX{} is a \enquote{mark-up} language, very much like HTML.

\subsection{¿Qué es un Acuario?}

	\begin{figure}[]
		\centering
	    \includegraphics[width=.5\textwidth]{./Figures/acuarioHobby3.jpg}
		\label{fig.acuarioHobby}
		\caption{Acuario}
	\end{figure}       

\subsection{¿Qué es el Proyecto CIAA?}

El Proyecto CIAA nació en 2013 como una iniciativa conjunta entre el sector académico y el industrial, representados por la ACSE y CADIEEL, respectivamente.
Objetivos del proyecto

*Impulsar el desarrollo tecnológico nacional, a partir de sumar valor agregado al trabajo y a los productos y servicios, mediante el uso de sistemas electrónicos, en el marco de la vinculación de las instituciones educativas y el sistema científico-tecnológico con la industria.
*Darle visibilidad positiva a la electrónica argentina.
*Generar cambios estructurales en la forma en la que se desarrollan y utilizan en nuestro país los conocimientos en el ámbito de la electrónica y de las instituciones y empresas que hacen uso de ella.

Crear un ecosistema de valor centrado en el trabajo colaborativo, transdisciplinario y en red que promueva la innovación para crear, diseñar y desarrollar soluciones electrónicas en la industria.
Visión

Generar ámbitos de articulación entre las instituciones educativas, el sistema científico-tecnológico y las industrias para lograr estándares mundiales de competitividad que permitan ocupar una posición de liderazgo en materia de innovación y desarrollo en la industria electrónica.

Todo esto en el marco de un trabajo libre, colaborativo y articulado entre industria y academia. 

%----------------------------------------------------------------------------------------

\section{Motivación}


\subsection{Dimensionamiento del mercado}

\subsection{Soluciones existentes}

\subsection{Aplicabilidad a otras áreas}





%----------------------------------------------------------------------------------------
