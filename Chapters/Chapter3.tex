% Chapter Template

\chapter{Diseño e Implementación} % Main chapter title

\label{Chapter3} % Change X to a consecutive number; for referencing this chapter elsewhere, use \ref{ChapterX}

%----------------------------------------------------------------------------------------
%	SECTION 1
%----------------------------------------------------------------------------------------
\section{Hardware}
\label{sec:hard}
En el mercado se consiguen distintas soluciones que se ajustan a las necesidades de cada aplicación. 
 
\noindent El panel solar es de las siguientes características:
\begin{itemize}
\item Dimensiones: 304x352x22mm
\item Peso: 1.36Kg
\item Corriente: 0.58A
\item Tensión: 17.4V
\item Potencia: 10W
\end{itemize}

%-----------------------------------
%	SUBSECTION 1
%-----------------------------------
\subsection{Esquemático de extensión de circuito cargador}
\label{sec:extensión}

%-----------------------------------
%	SUBSECTION 2
%-----------------------------------
\subsection{Diagrama de conexión}
\label{sec:conexión}

%----------------------------------------------------------------------------------------
%	SECTION 2
%----------------------------------------------------------------------------------------

\section{Arquitectura}
\label{sec:arq}

%-----------------------------------
%	SUBSECTION 1
%-----------------------------------
\subsection{Diagrama en bloques}

Nunc posuere quam at lectus tristique eu ultrices augue venenatis. Vestibulum ante ipsum primis in faucibus orci luctus et ultrices posuere cubilia Curae; Aliquam erat volutpat. Vivamus sodales tortor eget quam adipiscing in vulputate ante ullamcorper. Sed eros ante, lacinia et sollicitudin et, aliquam sit amet augue. In hac habitasse platea dictumst.

%-----------------------------------
%	SUBSECTION 2
%-----------------------------------
\subsection{Arquitectura modular}

Módulos, modelo de capas. APIs, interfaces.

%----------------------------------------------------------------------------------------
%	SECTION 3
%----------------------------------------------------------------------------------------
\section{Firmware}
\label{sec:firm}

\begin{itemize}
	\item Implementación de funcionalidades en la subcapa MAC

	\item Implementación de funcionalidades en la subcapa PHY

	\item Implementación de funcionalidades a nivel de aplicación
	
\end{itemize}

%-----------------------------------
%	SUBSECTION 1
%-----------------------------------
\subsection{Ajustes a Microstack}


%-----------------------------------
%	SUBSECTION 2
%-----------------------------------
\subsection{Diagramas de topologías implementadas}

\begin{figure}[h!]
	\centering
    \includegraphics[width=.8\textwidth]{./Figures/topologia.jpg}
	\label{fig:mote}
	\caption{1) Topología estrella. 2)Topología Peer-to-Peer}
\end{figure}

\begin{figure}[h!]
	\centering
    \includegraphics[width=.8\textwidth]{./Figures/cluster.jpg}
	\label{fig:mote}
	\caption{Topología árbol de cluster}
\end{figure}

%-----------------------------------
%	SUBSECTION 3
%-----------------------------------
\subsection{Principales funciones}

Funciones

