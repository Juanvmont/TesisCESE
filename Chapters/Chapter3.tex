% Chapter Template

\chapter{Diseño e Implementación} % Main chapter title

\label{Chapter3} % Change X to a consecutive number; for referencing this chapter elsewhere, use \ref{ChapterX}

%----------------------------------------------------------------------------------------
%	SECTION 1
%----------------------------------------------------------------------------------------
\section{Hardware}
\label{sec:hard}
En el mercado se consiguen distintas soluciones que se ajustan a las necesidades de cada aplicación. El sistema fotovoltaico propuesto originalmente estaba basado en módulos de voltaje de entre 0V y 12V y 0.5A, y un regulador de voltaje a la salida. Entre las interfaces para el usuario del nodo \textit{Mote LSE}, esta el puerto USB  el cual alimenta al circuito de control de carga bq24080 1-A con 5V(DC) a través de conector USB micro-B, para la carga de la batería de Li-ion de 3.7V y 900mAh.

Como medida de mitigación de riesgos, se planteó emplear componentes disponibles en el mercado local, 

 \begin{figure}[h!]
	\centering
    \includegraphics[width=1\textwidth]{./Figures/ks10t.jpg}
	\label{fig:GANTT2}
	\caption{Modulo Fotovoltaico policristalino de alto rendimiento KS10T}
\end{figure}

\noindent El panel solar es de las siguientes características:
\begin{itemize}
\item Dimensiones: 304x352x22mm
\item Peso: 1.36Kg
\item Corriente: 0.58A
\item Tensión: 17.4V
\item Potencia nominal: 10Wp
\end{itemize}

\begin{figure}[h!]
	\centering
    \includegraphics[width=1\textwidth]{./Figures/mecanicas.JPG}
	\label{fig:GANTT2}
	\caption{Características mecánicas}
\end{figure}

\begin{figure}[h!]
	\centering
    \includegraphics[width=0.7\textwidth]{./Figures/curva.JPG}
	\label{fig:GANTT3}
	\caption{Características eléctricas}
\end{figure}
%-----------------------------------
%	SUBSECTION 1
%-----------------------------------
\subsection{Esquemático de extensión de circuito cargador}
\label{sec:extensión}
\begin{figure}[h!]
	\centering
    \includegraphics[width=0.7\textwidth]{./Figures/circuito.jpg}
	\label{fig:GANTT3}
	\caption{Extensión de circuito cargador diseñado}
\end{figure}
%-----------------------------------
%	SUBSECTION 2
%-----------------------------------
\subsection{Diagrama de conexión}
\label{sec:conexión}

%----------------------------------------------------------------------------------------
%	SECTION 2
%----------------------------------------------------------------------------------------

\section{Arquitectura}
\label{sec:arq}

%-----------------------------------
%	SUBSECTION 1
%-----------------------------------
\subsection{Diagrama en bloques}

Nunc posuere quam at lectus tristique eu ultrices augue venenatis. Vestibulum ante ipsum primis in faucibus orci luctus et ultrices posuere cubilia Curae; Aliquam erat volutpat. Vivamus sodales tortor eget quam adipiscing in vulputate ante ullamcorper. Sed eros ante, lacinia et sollicitudin et, aliquam sit amet augue. In hac habitasse platea dictumst.

%-----------------------------------
%	SUBSECTION 2
%-----------------------------------
\subsection{Arquitectura modular}

Módulos, modelo de capas. APIs, interfaces.

%----------------------------------------------------------------------------------------
%	SECTION 3
%----------------------------------------------------------------------------------------
\section{Firmware}
\label{sec:firm}

\begin{itemize}
	\item Implementación de funcionalidades en la subcapa MAC

	\item Implementación de funcionalidades en la subcapa PHY

	\item Implementación de funcionalidades a nivel de aplicación
	
\end{itemize}

%-----------------------------------
%	SUBSECTION 1
%-----------------------------------
\subsection{Ajustes a Microstack}


%-----------------------------------
%	SUBSECTION 2
%-----------------------------------
\subsection{Diagramas de topologías implementadas}

\begin{figure}[h!]
	\centering
    \includegraphics[width=.8\textwidth]{./Figures/topologia.jpg}
	\label{fig:topo}
	\caption{1) Topología estrella. 2)Topología Peer-to-Peer}
\end{figure}

\begin{figure}[h!]
	\centering
    \includegraphics[width=.8\textwidth]{./Figures/cluster.jpg}
	\label{fig:clust}
	\caption{Topología árbol de cluster}
\end{figure}

%-----------------------------------
%	SUBSECTION 3
%-----------------------------------
\subsection{Principales funciones}

Funciones

