% Chapter Template

\chapter{Ensayos y Resultados} % Main chapter title

\label{Chapter4} % Change X to a consecutive number; for referencing this chapter elsewhere, use \ref{ChapterX}

%----------------------------------------------------------------------------------------
%	SECTION 1
%----------------------------------------------------------------------------------------

\section{Ensayos}

Para verificar y validar la correcta implementación del sistema se procede a ensayar el funcionamiento aplicando técnicas de ensayos de caja blanca y caja negra.
%-----------------------------------
%	SUBSECTION 1
%-----------------------------------
\subsection{Ensayo de caja blanca}



%-----------------------------------
%	SUBSECTION 2
%-----------------------------------

\subsection{Ensayo de caja negra}

Se utiliza la técnica de inyección de fallas para evaluar la respuesta del sistema frente a las condiciones de alarma.  El ensayo consiste en forzar las entradas de los sensores a valores que disparen las condiciones de alarma y contrastar las acciones de control del sistema con las esperadas.

Si bien el sistema tiene seis condiciones de alarma, cada uno de los tres sensores del sistema tiene dos condiciones, una por exceso y otra por defecto que son mutuamente excluyentes.  Por este motivo, el máximo número de alarmas simultáneas que puede haber en un momento dado es tres.  Para verificar el correcto comportamiento del sistema se elaboran tablas de decisión con las acciones que se deben tomar para las distintas combinaciones de alarmas. En el cuadro \ref{tab:1alarma} se indican las acciones de contingencia para cuando hay una única condición de alarma en el sistema.  En el cuadro \ref{tab:2alarmas} se indican las acciones para las distintas combinaciones de dos alarmas. En el cuadro \ref{tab:3alarmas} se indican las acciones de contingencias para las combinaciones de 3 alarmas.

\begin{table}[ht!]
	\centering
    \includegraphics[width=.7\textwidth]{./Figures/tabla1alarma.pdf}
	\caption{Tabla de decisión para el control de 1 sóla alarma.}
	\label{tab:1alarma}
\end{table}

\begin{table}[ht!]
	\centering
    \includegraphics[width=.9\textwidth]{./Figures/tabla2alarmas.pdf}
	\caption{Tabla de decisión para el control de 2 alarmas.}
	\label{tab:2alarmas}
\end{table}

\begin{table}[ht!]
	\centering
    \includegraphics[width=.8\textwidth]{./Figures/tabla3alarmas.pdf}
	\caption{Tabla de decisión para el control de 3 alarmas.}
	\label{tab:3alarmas}
\end{table}

%----------------------------------------------------------------------------------------
%	SECTION 2
%----------------------------------------------------------------------------------------

\section{Resultados}

\begin{figure}
\centering
    \includegraphics[width=\textwidth]{./Figures/Alarma1Nivel.pdf}
	\caption{Respuesta del sistema frente a variaciones en el nivel de agua.}
	\label{fig:alarma1Nivel}
\end{figure}

\begin{figure}
\centering
    \includegraphics[width=\textwidth]{./Figures/Alarma1Temp.pdf}
	\caption{Respuesta del sistema frente a variaciones en la temperatura.}
	\label{fig:alarma1Temp}
\end{figure}