% Chapter Template

\chapter{Ensayos y Resultados} % Main chapter title
\label{Chapter4} 

%Como crítica empírica se mostrarán la consistencias o inconsistencias de la teoría.
%Simulation is used when uncertainty of outcomes is high. La mayor desventaja de una simulación es que solo simula el procesador y la interacción con perifericos en un sistema embebido, a veces puede ser imitado con scripts de simulación y otras soluciones , pero a menudo estas soluciones son mas problemáticas de crear que el valor de la simulación.


%----------------------------------------------------------------------------------------
%	SECTION 1
%----------------------------------------------------------------------------------------
\section{Ensayo funcional}
\label{sec:funcional}

Se recurre a la metodología empírica analítica, o contraste empírico basado en la experimentación de la solución que se implementa en un despliegue real de red WSN.

El análisis comprende resultados de contraste con la respuesta esperada por la extensión del circuito cargador y la batería y de prueba experimental lógica argumentativa. Para verificar los resultados obtenidos y comprobar la respuesta esperada por cada nodo.

En la I Etapa de experimentación, la implementación consta de una red WSN con topología de punto a punto y de estrella. Además se comprueba que los resultados son consistentes a medida que crece la red.

Alcance? (con visión directa en condiciones indoor/outdoor).


\begin{table}[ht]
	\centering
	\caption{Asignación de puertos bq24080 - LPC1343}
	\begin{tabular}{@{} l *3c @{}}    \toprule
		\emph{\textbf{Puerto bq24080}} & \emph{\textbf{Puerto LPC1343}} & \emph{\textbf{PIN LPC1343}} & \emph{\textbf{PIN bq24080}}\\
		\midrule
		CE &  PIO3\_ 0 & 36 & 9	\\	
		PG	&  PIO3\_ 1 & 37 & 8\\
		STAT1 &  PIO1\_ 4 & 40 & 3\\
		STAT2 &  PIO2\_ 3 & 38 & 4\\
		\bottomrule
		\hline
	\end{tabular}
	\label{tab:bq}
\end{table}

%Por esta razon, la energia que provee el panel solar es ensayada como fuente de voltaje de 12V activa entre las 9 y 17 hs.

%No hay manera de asegurarse que los procedimientos de pruebas son una comparación acertada o representa la realidad, pero estos cumplen con sus características determinísticas.

%----------------------------------------------------------------------------------------
%	SECTION 2
%----------------------------------------------------------------------------------------
\section{Medición de consumos}
\label{sec:Medición}

\begin{table}[ht]
	\centering
	\caption{Medición de consumo según estado del circuito}
	\begin{tabular}{@{} l *1c @{}}    \toprule
		\emph{\textbf{Estado}} & \emph{\textbf{Consumo}}\\
		\midrule
		Rx &  PIO3\\	
		Tx	&  PIO3\\
		Sleep (Transceptor) &  PIO1\\
		Sleep (Controlador de carga) &  PIO2\\
		Sleep (Host) &  PIO2\\
		Máximo &  PIO2\\
		Mínimo &  PIO2\\
		\bottomrule
		\hline
	\end{tabular}
	\label{tab:bq}
\end{table}

%transceptor
%RX: 18.5 - 22.3 mA.
%TX: 25.8 - 33.6 mA.
%Tx 162mA @22dBm Pout
%Rx 39mA @-104dBm Sensibilidad



%\begin{table}[ht]
%	\centering
%	\caption{Contraste de mediciones del firmware vs testigo}
%	\begin{tabular}{@{} l *3c @{}}    \toprule
%		\emph{\textbf{Puerto bq24080}} & \emph{\textbf{Puerto LPC1343}} & \emph{\textbf{PIN LPC1343}} & \emph{\textbf{PIN bq24080}}\\
%		\midrule
%		Temperatura ambiente &  36 & 9	\\	
%		Voltaje de la batería	&  37 & 8\\
%		\bottomrule
%		\hline
%	\end{tabular}
%	\label{tab:bq}
%\end{table}

%----------------------------------------------------------------------------------------
%	SECTION 3
%----------------------------------------------------------------------------------------
\section{Análisis del tiempo de recarga}
\label{sec:recarga}
Gráfica tension/tiempo y estados superpuestos

%----------------------------------------------------------------------------------------
%	SECTION 4
%----------------------------------------------------------------------------------------
\section{Análisis del tiempo de vida}
\label{sec:vida}
Gráfica tension/tiempo en descarga

%----------------------------------------------------------------------------------------
%	SECTION 5
%----------------------------------------------------------------------------------------
\section{Estado del sistema}
\label{sec:estado}
breakpoints


