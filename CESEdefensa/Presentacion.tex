\documentclass[aspectratio=169, handout]{beamer}
%la opcion hangout es para complilar en modo imprimible
%\documentclass[hangout]{beamer}

\mode<presentation>
{
  \usetheme{Berkeley}
  \setbeamercovered{transparent}
  \setbeamertemplate{navigation symbols}{}
}

\usepackage[spanish]{babel}
\usepackage[utf8]{inputenc}
\usepackage{tikz}
\usepackage{textpos}
\usepackage{hyperref}
\usepackage{caption}
\captionsetup[figure]{labelformat=empty}

%\usetikzlibrary{shapes,arrows}
\setbeamerfont{author}{size=\large}
\setbeamerfont{institute}{size=\normalsize\bfseries}
\setbeamerfont{title}{size=\Large\bfseries}
\setbeamerfont{subtitle}{size=\huge}

\definecolor{darkblue}{RGB}{51,51,179}
\setbeamercolor{bgcolor}{fg=white,bg=darkblue}

%\title[WSN]{Sistema de Monitoreo de Salud de Nodos WSN Alimentados a Energía Solar}
\subtitle{WSN}
\author[]{Esp. Ing. Juan Montilla}
\institute[CESE-FIUBA]{Carrera de Especialización en Sistemas Embebidos - Facultad de Ingeniería - Universidad de Buenos Aires}
\date{}

%\subtitle{Framework para aplicaciones de control de ambientes}
\titlegraphic{\includegraphics[width=5cm]{./imagenes/red.jpg}}


\subject{Sistema de Monitoreo de Salud de Nodos WSN Alimentados a Energía Solar. Carrera de Especialización en Sistemas Embebidos}
% This is only inserted into the PDF information catalog. Can be left
% out. 

\pgfdeclareimage[height=1.5cm]{university-logo}{./imagenes/logo-facu-inverso.png}
\logo{\pgfuseimage{university-logo}}


% If you wish to uncover everything in a step-wise fashion, uncomment
% the following command: 

\beamerdefaultoverlayspecification{<+->}
  
\begin{document}

%Captions sin el texto "Figure"
\setbeamertemplate{caption}{\raggedright\insertcaption\par}

%la magia del begingroup es para que titlepage quede centrada, sin eso queda
%corrida en el ancho del sidebar
%\begingroup
%\makeatletter
%\setlength{\hoffset}{-.5\beamer@sidebarwidth}
%\makeatother
%\begin{frame}[plain,noframenumbering]
%  \titlepage
%\end{frame}
%
%\endgroup


%-------------------------------------------------%
%-------------------------------------------------%
% PORTADA
%-------------------------------------------------%
%-------------------------------------------------%

\begingroup
\makeatletter
\setlength{\hoffset}{-.5\beamer@sidebarwidth}
\makeatother
\begin{frame}[plain,noframenumbering]
\begin{center}
%\vspace{5px}
\hfill
    \begin{beamercolorbox}[center,dp=3ex,ht=10.25ex, wd=1\linewidth]{bgcolor}
        \Large\textbf{Sistema de Monitoreo de Salud de Nodos WSN Alimentados a Energía Solar}\\
       % \huge\textbf{802.15.4 LR-WPAN}
    \end{beamercolorbox}
\hfill\hfill
\\
\vspace{5px}
\textbf{Carrera de Especialización en Sistemas Embebidos}\\
\texttt{Facultad de Ingeniería - Universidad de Buenos Aires}\\
\vspace{10px}
\texttt{Esp. Ing. Juan Montilla}\\

\vspace{10px}

\begin{figure}[H]
	\includegraphics[width=.3\textwidth]{./imagenes/red.jpg}
\end{figure}	  	  	
\vspace{5px}
\tiny 01-08-2016 

\end{center}
\end{frame}
\endgroup

%-------------------------------------------------%
%-------------------------------------------------%

\begin{frame}{\textbf{Organización de la presentación}}
  \tableofcontents
  % You might wish to add the option [pausesections]
\end{frame}
%
%

%-------------------------------------------------%
%-------------------------------------------------%
\section{Introducción General}
%-------------------------------------------------%
%-------------------------------------------------%

%-------------------------------------------------%
\subsection[Motivación]{Motivación}
%-------------------------------------------------%

\begin{frame}{Motivación}{Motivación}
	\begin{itemize}
		\item Dispositivos con fuente de alimentación autónoma.
		\vspace{5px}
		\item Se presenta un problema de autonomía/vida útil.
		\vspace{5px}
		\item Brindar la posibilidad de detectarlo.
		\vspace{5px}
		\item Soluciones:
		\begin{itemize}
			\item Módulo fotovoltaico.
			\item Control de carga.
			\item Optimizar la vida útil de la batería.
			\item Extremadamente bajo consumo de potencia.
		\end{itemize}
	\end{itemize}

%Es util y anda bien
%Servicio a la bateria, mantenimiento
%	
\end{frame}

%-------------------------------------------------%
\subsection[WSN]{¿Qué es WSN?}
%-------------------------------------------------%
\begin{frame}{¿Qué es WSN?} 
\noindent Wireless Sensor Networks: \textbf{Redes de Sensores Inalámbricos}.
\begin{minipage}[c]{1.0\linewidth}
	\begin{minipage}[c]{0.5\linewidth}

		\begin{itemize}
			\item Medición inteligente.
	%	\item Comunicaciones y Control de ferrocarril
	%	\item Redes de monitoreo de infraestructuras criticas
			\item Domótica y seguridad.
			\item Productos electrónicos de consumo.
			\item Cuidado de la salud.
			\item Control y monitoreo de vehículos.
			\item Agricultura.
			\item Comunicación Militar.
		\end{itemize}

\end{minipage}
	\begin{minipage}[c]{0.45\linewidth}
		\begin{figure}[H]			
		\includegraphics[width=1.2\textwidth]{./imagenes/WSN.jpg}
		\end{figure}	  	  	
	\end{minipage}
\end{minipage}
\end{frame}

\begin{frame}{¿Qué es WSN?} 
\noindent Red Distribuida de WSN con interfaz de usuario a través de un sumidero.
		\begin{figure}[H]			
		\includegraphics[width=0.37\textwidth]{./imagenes/RedDistribuida.png}
		\end{figure}	  	  	
\end{frame}
%-------------------------------------------------%
\subsection[Energía]{Implicaciones de Energía}
%-------------------------------------------------%
\begin{frame}{Implicaciones de energía} 
\begin{minipage}[c]{1.0\linewidth}
	\begin{minipage}[c]{0.7\linewidth}
		\begin{itemize}
			\item El modulo fotovoltaico + pocos componentes.
					\vspace{10px}
			\item Generadores Eléctricos Solares Autónomos (GESA).
					\vspace{10px}
			\item Acceso reducido a Red de distribución eléctrica.
					\vspace{10px}
		\end{itemize}

\end{minipage}
	\begin{minipage}[c]{0.25\linewidth}
		\begin{figure}[H]			
		\includegraphics[width=1.2\textwidth]{./imagenes/ks10t.jpg}
		\end{figure}	  	  	
	\end{minipage}
\end{minipage}
\end{frame}
%-------------------------------------------------%
%-------------------------------------------------%
\section{Introducción Específica}
%-------------------------------------------------%
%-------------------------------------------------%
%-------------------------------------------------%
\subsection[Objetivos]{Descripción del trabajo}
%-------------------------------------------------%
\begin{frame}{Objetivos}{Descripción del trabajo}
% El primer minipage es un marco para las otras dos, que parten la pantalla en dos horizontalmente.
% Con 0.6\linewidth le indicás que porcentaje del ancho de la página debe tener la minipage
\noindent Los objetivos del trabajo son los siguientes:
\begin{itemize}
%	\item Full-function device (FFD):\\ Capaz de ser PAN coordinator o coordinator. 
%	\vspace{10px}

	\item Desarrollar un software de supervisión de nodos Mote LSE con extensión de panel solar.
	\item Implementar el sistema en n nodos desplegados en una red inalámbrica de área personal.
	\item Gestionar el modo de operación del nodo en función de la tensión que entrega el circuito cargador y la proyección de la batería restante.
	\item Supervisar la temperatura del entorno del nodo.
	\item Reportar a un nodo central el estado de salud del nodo.
	\item Fijar y leer en forma remota las alarmas/parámetros de configuración.
	 \end{itemize}	
\end{frame}

%-------------------------------------------------%
\subsection[Herramientas]{Herramientas}
%-------------------------------------------------%
\begin{frame}{Nodo Mote LSE-FIUBA} 

\begin{minipage}[c]{1.0\linewidth}
	\begin{minipage}[c]{0.6\linewidth}
\begin{itemize}
\item Microcontrolador NXP LPC1343.
   \begin{itemize}
   \item Procesador ARM Cortex-M3 de 32 bits @72MHz. 
   \item 32kB de memoria Flash.
   \item 8kB de memoria SRAM.
   \end{itemize}
\item Transceptor TI-2520 + Extensor TI-2591.
\item Controlador de carga bq24080.
\item Batería de Li-ion de 3.7V y 900mAh.
\item Sensores de luz y temperatura.
\item Antena y balun en microstrip.
\end{itemize}
	\end{minipage}
	\begin{minipage}[c]{0.35\linewidth}
		\begin{figure}[H]
			\vspace{35px}
			\includegraphics[width=1.2\textwidth]{./imagenes/mote.jpg}
			\\
			\vspace{10px}
			\includegraphics[width=1.2\textwidth]{./imagenes/motePCB}
			\label{Mote LSE}
			%\caption{Mote LSE}
		\end{figure}	  	  	
	\end{minipage}
\end{minipage}
\end{frame}

%--------------------------------------------------------------------%
\subsection[Plan]{Planificación}
%--------------------------------------------------------------------%
\begin{frame}[t]{Planificación - AON}
		\begin{figure}[H]
			{\includegraphics[width=.73\textwidth]{./imagenes/AON.PNG}}
		\end{figure}	 
\end{frame}

%-------------------------------------------------%
%-------------------------------------------------%
\section{Diseño e Implementación}
%-------------------------------------------------%
%-------------------------------------------------%
%-------------------------------------------------%
\subsection[Hardware]{Hardware}
%-------------------------------------------------%

%-------------------------------------------------%
\subsection[Arquitectura]{Arquitectura de Software}
%-------------------------------------------------%
\begin{frame}{Arquitectura del estándar}

\begin{minipage}[c]{1.0\linewidth}
	\begin{minipage}[c]{0.6\linewidth}
		\begin{itemize}
			\item MAC Sublayer
			\begin{itemize}
				\item Beacon management
				\item Channel access
				\item GTSs management
				\item Frame validation, ACKs
				\item Asociación y desasociación de dispositivos
			\end{itemize}
			\vspace{10px}
			\item Physical Layer (PHY):
			\begin{itemize}
				\item Activación/Desactivación de RF
				\item ED, LQI, Clear Channel Assessment (CCA)
				\item Channel selection
				\item Tx y Rx de paquetes a través del medio físico
			\end{itemize}
			\vspace{10px}
		\end{itemize}	
	  \end{minipage}
	  \begin{minipage}[c]{0.35\linewidth}
		\begin{figure}[H]
			{\includegraphics[width=.6\textwidth]{./imagenes/arquitectura}}
		\end{figure}	  	  	
	  \end{minipage}
\end{minipage}

\end{frame}

%-------------------------------------------------%
\subsection[Firmware]{Firmware}
%-------------------------------------------------%
\begin{frame}{Topología de la Red}{Estrella o punto a punto}

\begin{minipage}[c]{1.0\linewidth}
	\begin{minipage}[c]{0.45\linewidth}
		\begin{itemize}
			\item Estrella (Star)
			\begin{itemize}
				\item PAN coordinator.
				\item Comunicaciones centralizadas.
				\item Ej: Automatización del hogar, Periféricos de PC, Juegos,...
					\end{itemize}
			\vspace{10px}
			\item Punto a punto (Peer-to-Peer)
			\begin{itemize}
				\item PAN coorditator.
				\item Permite redes más complejas.
				\item Multi-Hop routing.
				\item Ej: Control industrial,  WSNs, Tracking de inventario,...
			\end{itemize}
	  	\end{itemize}	
	\end{minipage}
	\hspace{-20px}
	\begin{minipage}[c]{0.7\linewidth}
		\begin{figure}[H]
			{\includegraphics[width=.7\textwidth]{./imagenes/Topology}}
		\end{figure}	  	  	
	\end{minipage}
\end{minipage}
\end{frame}

\begin{frame}{Topología Punto a punto}{Árbol de Cluster}
\begin{minipage}[c]{1.0\linewidth}
\begin{minipage}[c]{0.45\linewidth}
		\begin{itemize}
			\item Mayoría de FFDs.
			\vspace{10px}
			\item 1 \textit{overall PAN coordinator}.
			\vspace{10px}
			\item RFDs al final de una rama.
			\vspace{10px}
			\item Aumenta el área de covertura.
			\vspace{10px}
			\item Aumenta la latencia de la red.
		\end{itemize}	
	\end{minipage}
	\hspace{-15px}
	\begin{minipage}[c]{0.65\linewidth}
		\begin{figure}[H]
			{\includegraphics[width=.8\textwidth]{./imagenes/cluster}}
		\end{figure}	  	  	
	\end{minipage}
\end{minipage}
\end{frame}

%-------------------------------------------------%
%-------------------------------------------------%
\section{Ensayos y Resultados}
%-------------------------------------------------%
%-------------------------------------------------%


%-------------------------------------------------%
%-------------------------------------------------%
\section{Trabajos Futuros}
%-------------------------------------------------%
%-------------------------------------------------%


%%-------------------------------------------------%
%%-------------------------------------------------%
%\section{Referencias}
%%-------------------------------------------------%
%%-------------------------------------------------%
%
%\begin{frame}[c]{Referencias}
%
%%\Large{Referencias}
%\vspace{20px}
%\begin{itemize}
%	\item<.-> \href{http://ecee.colorado.edu/~liue/teaching/comm_standards/2015S_zigbee/802.15.4-2011.pdf}{Estándar IEEE 802.15.4:2011}
%	\vspace{5px}	
%	\item<.-> \href{http://www.ieee802.org/15/pub/TG4.html}{IEEE 802.15 - Task Group 4 - Home Page}
%	\vspace{5px}
%	\item<.-> \href{	https://standards.ieee.org/about/get/802/802.15.html}{IEEE Get Program}
%	\vspace{5px}
%	\item<.-> \href{http://www.nxp.com/documents/data_sheet/LPC1311_13_42_43.pdf}{LPC1343 - Datasheet}
%	\vspace{5px}
%	\item<.-> \href{http://www.nxp.com/documents/user_manual/UM10375.pdf}{LPC1343 - User Manual}
%	\vspace{5px}
%	\item<.-> \href{http://www.ti.com/product/CC2520/technicaldocuments}{Texas Instrument CC2520 -  Technical Documents}
%	\vspace{5px}
%	\item<.-> \href{http://www.ti.com/lit/an/swru120b/swru120b.pdf}{Texas Instrument Design Note - 2.4 GHz Inverted F Antenna}
%\end{itemize}
%\end{frame}

%\section*{cartón de gracias}

\begingroup
\makeatletter
\setlength{\hoffset}{-.5\beamer@sidebarwidth}
\makeatother
\begin{frame}[plain,noframenumbering]
\begin{center}
%\vspace{5px}
\hfill
    \begin{beamercolorbox}[center,dp=3ex,ht=10.25ex, wd=1\linewidth]{bgcolor}
        \Large\textbf{Sistema de Monitoreo de Salud de Nodos WSN Alimentados a Energía Solar}\\
    \end{beamercolorbox}
\hfill\hfill
\\
\vspace{5px}
\textbf{Carrera de Especialización en Sistemas Embebidos}\\
\textbf{Facultad de Ingeniería - Universidad de Buenos Aires}\\
\vspace{10px}
\texttt{Esp. Ing. Juan V. Montilla C.}\\

\vspace{10px}

\begin{figure}[H]
	\includegraphics[width=.3\textwidth]{./imagenes/red.jpg}
\end{figure}	

\vspace{5px}
\tiny versión: 01-08-2016 
 	  	
\end{center}
\end{frame}
\endgroup

\end{document}
